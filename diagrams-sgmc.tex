% NAME=streaming-wm && time convert -density 300 -delay 15 -loop 0 $NAME.pdf $NAME.gif && convert "$NAME.gif[-1]" $NAME-final.png

\documentclass[usenames,dvipsnames,tikz,convert={outfile=\jobname.png}]{standalone}
\usepackage{graphicx}
\usepackage{balance}  % for  \balance command ON LAST PAGE  (only there!)
\usepackage{color}
\usepackage{listings}
\usepackage{courier}
%\usepackage[usenames,dvipsnames]{xcolor}
\usepackage{textgreek}
\usepackage{verbatim}
\usepackage{setspace}
\usepackage[outline]{contour}
\usepackage{lmodern}
\usepackage{ifthen}

% Roboto
% http://www.tug.dk/FontCatalogue/robotoregular/
%\usepackage[sfdefault]{roboto}  %% Option 'sfdefault' only if the base font of the document is to be sans serif
%\usepackage[T1]{fontenc}

% UWR Nimbus Sans
% http://www.tug.dk/FontCatalogue/urwnimbussans/
%\usepackage[scaled]{helvet}
%\renewcommand*\familydefault{\sfdefault} %% Only if the base font of the document is to be sans serif
%\usepackage[T1]{fontenc}

% Fira Sans
% http://www.tug.dk/FontCatalogue/firasansnewtxsf/
\usepackage[T1]{fontenc}
\usepackage[sfdefault,scaled=.85]{FiraSans}
\usepackage{newtxsf}

\SetSymbolFont{letters}{bold}{OML}{cmbr}{bx}{it}
\renewcommand{\familydefault}{\sfdefault}

\usepackage{tikz}
\usetikzlibrary{calc} % for let
\usetikzlibrary{shapes,snakes} % for let
\usetikzlibrary{decorations.pathreplacing}
\usetikzlibrary{patterns}
\usetikzlibrary{backgrounds}
\usepackage[active,tightpage]{preview}

\begin{document}

\definecolor{c_back}{HTML}{FFFFFF}
\definecolor{c_datum_back}{HTML}{7BCFA9}%E57368}
\definecolor{c_datum_back_ghost}{HTML}{DDDDDD}%E57368}
\definecolor{c_datum_text}{rgb}{0, 0, 0}
\definecolor{c_state_back}{HTML}{999999} %7BCFA9} 
\definecolor{c_state_text}{rgb}{0, 0, 0}
%\definecolor{c_state_text}{rgb}{255, 255, 255}
\definecolor{c_out_back}{HTML}{A0C3FF}%76A7FA}%FFFFFF} %A0C3FF}
\definecolor{c_out_text}{rgb}{0, 0, 0}
%\definecolor{c_out_text}{HTML}{A0C3FF}%76A7FA}%FFFFFF} %A0C3FF}
\definecolor{c_trigger}{HTML}{FFFFFF}%76A7FA}
\definecolor{c_retraction_back}{HTML}{E57368}%FFE168}
\definecolor{c_retraction_text}{rgb}{0, 0, 0}
\definecolor{c_timeline}{HTML}{000000}%76A7FA}
\definecolor{c_proc_time}{HTML}{76A7FA}
\definecolor{c_event_time}{HTML}{C8C843}
\definecolor{c_watermark}{HTML}{ED9D97}
\definecolor{c_ideal}{rgb}{.5, .5, .5}
\definecolor{c_boundary}{HTML}{76A7FA}%{rgb}{.7, .7, .5}
\definecolor{c_tomb}{HTML}{FFFFFF}%5D5D5D}
\definecolor{c_watermark_legend}{HTML}{000000}

% Opacities
\def \ostate{.111}%.333} %.7070}
\def \oout{.333} %0.796875
\def \oretraction{.333} % 0.75
\def \otrigger{.5}

\def \ofadeA{.1} %.2
\def \ofadeB{.05} %.1
\def \ofadeC{.01} 

\newcommand{\TikzDiagram}[2][8cm] {
\begin{tikzpicture}
[color=white,show background rectangle, background rectangle/.style={fill=c_back},font=\sffamily]
	#2
\end{tikzpicture}
\newpage
}

\newcommand{\WindowBrace}[5] {
	\draw [line width=1pt,gray] (#1, #2) -- (#1 - #4, #2);
	\draw [line width=1pt,gray] (#1, #3) -- (#1 - #4, #3);
	\draw [line width=1pt,gray] (#1 - #4, #2) -- (#1 - #4, #3);
	\draw [line width=1pt,gray] (#1 - #4, {#2 - (#2 - #3) / 2}) -- (#1 - #4 - #4, {#2 - (#2 - #3) / 2});
	\node [gray,font=\large] at (#1 - #4 - #4 - #4, {#2 - (#2 - #3) / 2}) {#5};
}


\tikzstyle{s_ideal}=[loosely dotted,line width=1.5pt,color=c_ideal]
\tikzstyle{s_watermark}=[line width=1.5pt, solid, color=c_watermark]
\tikzstyle{s_legend}=[font=\small]
\tikzstyle{s_timepoint}=[font=\small]
\tikzstyle{s_timeline}=[solid,line width=1.5pt,color=c_timeline]
\tikzstyle{s_boundary}=[color=c_boundary, densely dashed]
\tikzstyle{s_late}=[color=c_state_back,pattern=north west lines,pattern color=c_state_back]
\tikzstyle{s_tomb}=[color=white,densely dotted, line width=.5pt] 
\tikzstyle{s_tomb_late}=[color=white,densely dotted, line width=.5pt] 
%\tikzstyle{s_tomb_late}=[color=white,densely dotted,pattern=crosshatch dots,pattern color=c_tomb] 

% Default to input version #1
\ifdefined \InputVersion
\else
	\def \InputVersion{1}
\fi

\newcommand{\Version}[4] {
	\if \InputVersion 1
		#1
	\else
		\if \InputVersion 2
			#2
		\else
			\if \InputVersion 3
				#3
			\else
				#4
			\fi
		\fi
	\fi
}

\Version
  {\def \maxx{9}}
  {\def \maxx{9}}
  {\def \maxx{10}}
  {\def \maxx{10}}
  

\def \maxy{5}
\def \yoff{5}
\def \scale{.9}
\def \trioff{0.0265}

\newcommand {\DrawAxes}[0] {
	
	% X Axis
	\draw[color=c_event_time, line width=1.5pt] (0,0) -- (\maxx,0);
	% X Axis Tick Marks
	\foreach \x in {1,...,\maxx} {
		\draw[color=c_event_time] (\x, .1) -- (\x, -.1);
		\node[s_timepoint,color=c_event_time] at (\x, -.25) {12:0\x};
	}
	% X Axis Label
	\node[s_legend,color=c_event_time] at (\maxx / 2, -.666) {Event Time};

	% Y Axis
	\draw[c_proc_time, line width=1.5pt] (0,-0.0265) -- (0,\maxy);
	% Y Axis Tick Marks
	\foreach \y / \t in {1/06, 2/07, 3/08, 4/09, 5/10} {
		\draw[c_proc_time] (.1, \y) -- (-.1, \y);
		\node[s_timepoint, rotate=90, c_proc_time] at (-.25, \y) {12:\t};
	}
	% Y Axis Label
	\node[s_legend,rotate=90, c_proc_time] at (-.666, \maxy / 2) {Processing Time};	
	
	% Corner beautification
	\fill[c_event_time, line width=0pt] (-\trioff, -\trioff) -- (\trioff, \trioff) -- (\trioff, -\trioff) -- (-\trioff, -\trioff);
}

\newcommand {\DrawInput}[4][c_datum_back] {
	\filldraw[#1, draw=c_datum_text] (#2, #3) circle [radius=6pt];
	\node[c_datum_text,font=\bf\sffamily] at (#2, #3) {#4};
}

	% A inputs, in sorted order (by processing-time)
	%\DrawInput{0.444}{.333}{5}
	%\DrawInput{2.444}{.666}{7}
	%\DrawInput{3.666}{1.222}{3}
	%\DrawInput{3.888}{1.444}{4}
	%\DrawInput{4.333}{1.666}{3}
	%\DrawInput{3.111}{2.111}{8}
	%\DrawInput{6.666}{2.333}{3}
	%\DrawInput{1.444}{3.333}{9}
	%\DrawInput{7.444}{3.666}{8}
	%\DrawInput{7.777}{4.0}{1}

\newcommand {\DrawInputsA} {
	\DrawInput{0.444}{.333}{5}
	\DrawInput{2.444}{.666}{7}
	\DrawInput{3.111}{2.111}{8}
	\DrawInput{3.666}{1.222}{3}
	\DrawInput{3.888}{1.444}{4}
	\DrawInput{4.333}{1.666}{3}
	\DrawInput{6.666}{2.333}{3}
	\DrawInput{7.444}{3.666}{8}
	\DrawInput{7.777}{4.0}{1}
	\DrawInput{1.444}{3.333}{9}
}

	% B inputs, in sorted order (by processing-time)
	%\DrawInput[#1]{2.444}{.888}{7}
	%\DrawInput[#1]{3.666}{1.111}{3}
	%\DrawInput[#1]{1.444}{1.333}{9}
	%\DrawInput[#1]{3.111}{1.777}{8}
	%\DrawInput[#1]{6.666}{1.999}{3}
	%\DrawInput[#1]{0.444}{2.111}{5}
	%\DrawInput[#1]{7.444}{2.555}{8}
	%\DrawInput[#1]{3.888}{2.888}{4}
	%\DrawInput[#1]{7.777}{3.222}{1}
	%\DrawInput[#1]{4.333}{3.888}{3}

\newcommand {\DrawInputsB} {
	\DrawInput{0.444}{2.111}{5}
	\DrawInput{2.444}{.888}{7}
	\DrawInput{3.111}{1.777}{8}
	\DrawInput{3.666}{1.111}{3}
	\DrawInput{3.888}{2.888}{4}
	\DrawInput{4.333}{3.888}{3}
	\DrawInput{6.666}{1.999}{3}
	\DrawInput{7.444}{2.555}{8}
	\DrawInput{7.777}{3.222}{1}
	\DrawInput{1.444}{1.333}{9}
}

\tikzstyle{s_datum_ghost}=[color=c_datum_back_ghost,opacity=.333]

\newcommand{\DrawShiftedInput}[2] {
	\DrawInput{5 + #1}{#1}{#2}
}

\newdimen \deltaDim
\newdimen \offDim

\newcommand{\DrawIngressInput}[3] {
	\def \xL{#1}
	\def \xR{5 + #2}
	\def \arrowOff{.3}

	% Floating point calculations have to be done via
	% dimensions, which have to be set using evaluated
	% numbers, which appears to doable with \pgfmathparse
	% and \pgfmathresult...
	\pgfmathparse{\xR - \xL}
	\deltaDim=\pgfmathresult pt
	
	\pgfmathparse{\arrowOff * 2}
	\offDim=\pgfmathresult pt

	\DrawInput[s_datum_ghost]{\xL}{#2}{#3}
	\ifthenelse{\lengthtest{\deltaDim > \offDim}}
		{\draw[s_datum_ghost,-stealth,densely dotted,line width=.75pt] (\xL + \arrowOff, #2) -- (\xR - \arrowOff, #2);}
		{}
	\DrawInput{\xR}{#2}{#3}
}

\newcommand {\DrawInputsC} {
	\DrawIngressInput{0.444}{.333}{5}
	\DrawIngressInput{2.444}{.666}{7}
	\DrawIngressInput{3.111}{2.111}{8}
	\DrawIngressInput{3.666}{1.222}{3}
	\DrawIngressInput{3.888}{1.444}{4}
	\DrawIngressInput{4.333}{1.666}{3}
	\DrawIngressInput{6.666}{2.333}{3}
	\DrawIngressInput{7.444}{3.666}{8}
	\DrawIngressInput{7.777}{4.0}{1}
	\DrawIngressInput{1.444}{3.333}{9}
}

\newcommand {\DrawInputsD} {
	\DrawIngressInput{0.444}{2.111}{5}
	\DrawIngressInput{2.444}{.888}{7}
	\DrawIngressInput{3.111}{1.777}{8}
	\DrawIngressInput{3.666}{1.111}{3}
	\DrawIngressInput{3.888}{2.888}{4}
	\DrawIngressInput{4.333}{3.888}{3}
	\DrawIngressInput{6.666}{1.999}{3}
	\DrawIngressInput{7.444}{2.555}{8}
	\DrawIngressInput{7.777}{3.222}{1}
	\DrawIngressInput{1.444}{1.333}{9}
}

\newcommand {\DrawInputs} {
	\Version{\DrawInputsA}{\DrawInputsB}{\DrawInputsC}{\DrawInputsD}
}

\newcommand {\DrawMicroBatchBoundaries}[0] {
	\DrawBatchBoundary{1}
	%\DrawBatchBoundary{2}
	\DrawBatchBoundary{3}
	%\DrawBatchBoundary{4}
	\DrawBatchBoundary{5}
}

\newcommand{\DrawBatchBoundary}[1] {
	\draw[s_boundary] (0,#1) -- (\maxx,#1);
}

\newcommand {\DrawWatermarkA}[1][c_watermark] {
	\draw [-, s_watermark, #1] 
		(0.111,0.01)
		to [out=80,in=180] (0.444, .666) % 5
		to [out=0, in=180] (2.444, .999) % 7
		to [out=0, in=180] (3.111, 2.444) % 8
		to [out=0, in=180] (6.555, 2.666) % 3 (rightmost)
		to [out=0, in=180] (7.333, 4.000) % 8
		to [out=0, in=180] (7.7, 4.333) % 1
		to [out=0, in=180] (9.000, 4.37) % end
	;
}

\newcommand {\DrawWatermarkB}[1][c_watermark] {
	\draw [-, s_watermark, #1] 
		(0.111,0.01)
		to [out=80,in=180] (1.444, 1.666) % 9
		to [out=0, in=180] (3.111, 2.111) % 8
		to [out=0, in=180] (3.888, 3.222) % 4
		to [out=0, in=180] (7.333, 3.333) % 8
		to [out=0, in=180] (7.666, 3.555) % 1
		to [out=0, in=180] (9.000, 3.777) % end
	;
}

\newcommand{\DrawIdealWatermark}[1][s_ideal] {
	\draw [-,#1] (\yoff,0) -- (\maxx, \maxx - \yoff);
}

\newcommand{\DrawWatermark}[1][c_watermark] {
	\Version
	  {\DrawWatermarkA[#1]}
	  {\DrawWatermarkB[#1]}
	  {\DrawIdealWatermark[s_watermark]}
	  {\DrawIdealWatermark[s_watermark]}
}

\newcommand {\DrawBatchWatermarkAtTo}[3][c_watermark] {
	\draw [->, s_watermark, #1] (0, #2) -- (#3, #2 + #3 / 8 * .125);
}

\newcommand{\DrawTerminatedWatermark}[3][c_watermark] {
	\DrawBatchWatermarkAtTo[#1]{#2}{#3};
	\draw [#1,s_watermark, ->] (#3, #2 + #3 / 8 * .125) -- (8, #2 + #3 / 8 * .125);
}

\newcommand {\DrawBatchWatermarkAt}[2][c_watermark] {
	\DrawBatchWatermarkAtTo[#1]{#2}{8}
}

\newcommand {\DrawGlobalBatchWatermark}[1][c_watermark] {
	\DrawBatchWatermarkAt[#1]{4.25}
}

\newcommand {\DrawMicroBatchWatermarks}[1][c_watermark] {
	\DrawBatchWatermarkAt[#1]{4}
	\DrawBatchWatermarkAt[#1]{3}
	\DrawTerminatedWatermark[#1,-]{2}{7}
	\DrawTerminatedWatermark[#1,-]{1}{6}
}

\newcommand{\DrawWatermarks}[2][\DrawWatermark] {
	% Ideal watermark line
	\DrawIdealWatermark;
	
	% Watermark s_legends
	\node[s_legend, color=c_watermark_legend] at (2.5, -1.55) {Ideal watermark:};
	\draw [s_ideal, -] (4.5, -1.55) -- (6.5, -1.55);
	\node[s_legend,color=c_watermark_legend] at (2.5, -1.15) {Actual watermark:};
	\draw [s_watermark, -] (4.5, -1.15) -- (6.5, -1.15);

	#1
	
	% Blocker
	\fill[color=c_back,line width=.5cm] (0.01, 5) rectangle (\maxx, #2);

}

\contourlength{1pt}

\newcommand {\DrawCustomOutput}[8] {
	% arguments: 
	% #1 = x1
	% #2 = x2
	% #3 = y1
	% #4 = y2
	% #5 = text
	% #6 = box formatting
	% #7 = border + text formatting
	% #8 = opacity
	\fill[#6,#8,line width=.5cm] (#1, #3) rectangle (#2, #4);
	\draw[#6] (#1, #3) rectangle (#2, #4);
	\node[#7,font=\bf\sffamily\normalsize] at ({#1 + (#2 - #1) / 2}, {#4 - .25}) {#5};  %{\contour{white}{#5}}; 
}

\def \ShowLateness{0}
\def \latey{.1}
\newcommand{\DrawLatenessHorizonText}[3]{
		\node[font=\sffamily\tiny] at (#2 + \lateness, #3 +.4) {[12:0#1, 12:0#2)};
		\node[font=\sffamily\tiny] at (#2 + \lateness, #3 +.25) {Lateness Horizon};
}

\newdimen \lateXDim
\newdimen \maxXDim

\newcommand {\DrawState}[6][\ostate] {
	\DrawCustomOutput{#2}{#3}{#4}{#5}{#6}{c_state_back}{c_state_text}{opacity=#1} 
	\if \ShowLateness 1
	
		\pgfmathparse{#3 + \lateness}
		\lateXDim=\pgfmathresult pt
	
		\maxXDim=\maxx pt

		\ifthenelse{\lengthtest{\lateXDim > \maxXDim}}
			{\def \lateX{\maxx}}
			{\def \lateX{#3 + \lateness}}
	
		\draw [s_timeline] ({#3 + \lateness}, {#5 - \latey}) -- ({#3 + \lateness}, {#5 + \latey});
		\DrawLatenessHorizonText{#2}{#3}{#5}
		%\DrawCustomOutput{#3}{#3 + 1}{#4}{#5}{}{s_late}{c_state_text}{opacity=#1} 
	\fi
}

\newcommand {\DrawOutput}[6][c_out_back] {
	\DrawCustomOutput{#2}{#3}{#4}{#5}{#6}{#1}{c_out_text}{opacity=\oout} 
}

\newcommand {\DrawTrigger}[5] {
	\DrawCustomOutput{#1}{#2}{#3}{#4}{#5\phantom{---------------}}{white,densely dashed, line width=1pt,draw opacity=1}{white,text opacity=1}{opacity=0} 
}

\newcommand{\offx}[1]{#1 - .04}
\newcommand{\offy}[1]{#1 + .04}

\newcommand{\DrawRetraction}[5]{
	\DrawCustomOutput{#1}{#2}{#3}{#4}{-#5\phantom{-}}{c_retraction_back}{c_retraction_text}{opacity=\oretraction}
}

\newcommand{\DrawTombstone}[4]{
	\DrawCustomOutput{#1}{#2}{#3}{#4}{}{s_tomb}{}{opacity=0}
	\draw [s_tomb_late] (#2, #4) -- ({#2 + \lateness}, #4);
	\draw [s_tomb_late] ({#2 + \lateness}, {#4 - \latey}) -- ({#2 + \lateness}, {#4 + \latey});
	\DrawLatenessHorizonText{#1}{#2}{#4}
}

\newcommand{\DrawTimeline}[2]{
	\draw [s_timeline,#1] (0, #2) -- (\maxx, #2);
}

\def \ShowWatermarks{1}

\newcommand {\TriggersDiagram}[3][c_timeline] {
\TikzDiagram[7cm]{
	\if \ShowWatermarks 1
		\DrawWatermarks{#2} 
	\fi
	\DrawAxes 
	\DrawInputs
	#3
	\DrawTimeline{#1}{#2}
}
}

\newcommand{\DrawFinalFrames}[2][20] {
	\foreach \i in {1,...,#1} {
		#2
	}
}
