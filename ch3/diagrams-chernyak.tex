% NAME=streaming-wm && time convert -density 300 -delay 15 -loop 0 $NAME.pdf $NAME.gif && convert "$NAME.gif[-1]" $NAME-final.png

\documentclass[usenames,dvipsnames,tikz,convert={outfile=\jobname.png}]{standalone}
\usepackage{graphicx}
\usepackage{balance}  % for  \balance command ON LAST PAGE  (only there!)
\usepackage{color}
\usepackage{listings}
\usepackage{courier}
%\usepackage[usenames,dvipsnames]{xcolor}
\usepackage{textgreek}
\usepackage{verbatim}
\usepackage{setspace}
\usepackage[outline]{contour}
\usepackage{lmodern}
\usepackage{ifthen}
\usepackage{textcomp}
\usepackage{marvosym}
\usepackage{eurosym}

\def \ShowTiming{0}
\def \DisableTiming{0}

% Annoying global flag used to trigger ghost choices
\def \IsGhost{0}
\ifdefined \NineIsLate
\else
	\def \NineIsLate{0}
\fi
\ifdefined \UsePerfectWatermark
\else
	\def \UsePerfectWatermark{0}
\fi

\ifdefined \StreamsTables
\else
	\def \StreamsTables{0}
\fi

%\if \StreamsTables 0
	% Roboto
	% http://www.tug.dk/FontCatalogue/robotoregular/
%	\usepackage[sfdefault]{roboto}  %% Option 'sfdefault' only if the base font of the document is to be sans serif
%	\usepackage[T1]{fontenc}

% UWR Nimbus Sans
% http://www.tug.dk/FontCatalogue/urwnimbussans/
%\usepackage[scaled]{helvet}
%\renewcommand*\familydefault{\sfdefault} %% Only if the base font of the document is to be sans serif
%\usepackage[T1]{fontenc}

%\else
	% Fira Sans
	% http://www.tug.dk/FontCatalogue/firasansnewtxsf/
	\usepackage[T1]{fontenc}
	\usepackage[sfdefault,scaled=.85]{FiraSans}
	\usepackage{newtxsf}
%\fi

\SetSymbolFont{letters}{bold}{OML}{cmbr}{bx}{it}
\renewcommand{\familydefault}{\sfdefault}

\usepackage{tikz}
\usetikzlibrary{calc} % for let
\usetikzlibrary{shapes,snakes} % for let
\usetikzlibrary{decorations.pathreplacing}
\usetikzlibrary{patterns}
\usetikzlibrary{backgrounds}

\begin{document}

\ifdefined \ColorScheme
\else
	\def \ColorScheme{1} % light
\fi

\definecolor{c_grey}{HTML}{CCCCCC}

\ifdefined \OutputContour
\else
	\def \OutputContour{1pt}
\fi

\if \ColorScheme 1 % white
	\definecolor{c_blue_base}{HTML}{7788EE} %FF2222}  %8899FF %4285F4}
	\colorlet{c_blue}{c_blue_base!100!black}
	\definecolor{c_red_base}{HTML}{FF848F} %FF8888} %EA4335}
	\colorlet{c_red}{c_red_base!100!black}
	\definecolor{c_yellow}{HTML}{FFCC22} % FBB405
	\definecolor{c_green_base}{HTML}{33CC77} %34A853}
	\colorlet{c_green}{c_green_base!100!black}

	\definecolor{c_back_dark}{HTML}{FFFFFF}
	\definecolor{c_back}{HTML}{FFFFFF}

	\def \InputContour{0.65pt}
	\definecolor{c_datum_back}{HTML}{505050} % orig={7BCFA9} %E57368}
	\definecolor{c_datum_border}{HTML}{000000} % orig={7BCFA9} %E57368}
	\tikzstyle{s_datum_back}=[c_datum_back] 
	\definecolor{c_datum_back_ghost}{HTML}{CCCCCC}%E57368}
	\definecolor{c_datum_text}{HTML}{FFFFFF}
	\colorlet{c_state_back}{c_grey!60!black}%blue_light}

	\definecolor{c_timeline}{HTML}{000000} %888888}%76A7FA}
	\definecolor{c_ghost}{HTML}{CFCFCF} % BBBBBB
	\definecolor{c_ghost_text}{HTML}{EEEEEE} % DDDDDD
	\definecolor{c_ghost_border}{HTML}{999999}
	\definecolor{c_ghost_arrow}{HTML}{777777}
	\colorlet{c_legend}{black}
	\tikzstyle{s_dropped}=[red!80!white,draw opacity=1] 
	\definecolor{c_dropped_border}{HTML}{660000} 

	%\colorlet{c_outlined_text}{black}%white!25!black}
	%\definecolor{c_outlined_text_outline}{HTML}{EEEEEE}
	\colorlet{c_outlined_text}{white}
	\colorlet{c_outlined_text_outline}{white!25!black}
	\colorlet{c_tomb}{black}
	\definecolor{c_ideal}{rgb}{.8, .8, .8}
	
	\colorlet{c_bullet_draw}{black}
	\colorlet{c_bullet_fill}{white!25!black}
	
	\colorlet{c_toggle_a}{white}
	\colorlet{c_toggle_b}{RubineRed}
\else % dark
	\definecolor{c_blue}{HTML}{76A7FA}
	\definecolor{c_red}{HTML}{ED9D97}
	\definecolor{c_yellow}{HTML}{FBCB43}
	\definecolor{c_green}{HTML}{7BCFA9}

	\definecolor{c_back_dark}{HTML}{3D3D3D}
	\definecolor{c_back}{HTML}{5D5D5D}

	\def \InputContour{0pt}
	\definecolor{c_datum_back}{HTML}{FFFFFF} % orig={7BCFA9} %E57368}
	\definecolor{c_datum_border}{HTML}{000000}
	\tikzstyle{s_datum_back}=[c_datum_back] 
	\definecolor{c_datum_back_ghost}{HTML}{DDDDDD}%E57368}
	\definecolor{c_datum_text}{rgb}{0, 0, 0}
	\colorlet{c_state_back}{c_grey}%blue_light}

	\definecolor{c_timeline}{HTML}{FFFFFF}%76A7FA}
	\definecolor{c_ghost}{HTML}{666666}
	\definecolor{c_ghost_text}{HTML}{3A3A3A}
	\definecolor{c_ghost_border}{HTML}{3A3A3A}
	\colorlet{c_ghost_arrow}{c_datum_back_ghost}
	\colorlet{c_legend}{white}
	\tikzstyle{s_dropped}=[red!60!white,draw opacity=.25] 
	\definecolor{c_dropped_border}{HTML}{660000} 

	\colorlet{c_outlined_text}{white}
	\colorlet{c_outlined_text_outline}{white!25!black}
	\colorlet{c_tomb}{white}
	\definecolor{c_ideal}{rgb}{.5, .5, .5}

	\colorlet{c_bullet_draw}{white!25!black}
	\colorlet{c_bullet_fill}{white}
	
	\colorlet{c_toggle_a}{white}
	\colorlet{c_toggle_b}{violet!60!white}
\fi

\colorlet{c_blue_light}{c_blue!60!white}
\colorlet{c_yellow_light}{c_yellow!80!white}
\colorlet{c_yellow_dark}{c_yellow!80!black}
\definecolor{c_white}{HTML}{FFFFFF}

\definecolor{c_back_light}{HTML}{7D7D7D}
\colorlet{c_state_text}{c_white}%yellow_light}
\colorlet{c_out_back}{c_blue}
\colorlet{c_out_text}{c_yellow}

\definecolor{c_trigger}{HTML}{FFFFFF}%76A7FA}
\colorlet{c_retraction_back}{c_red}
\colorlet{c_retraction_text}{c_red}

\definecolor{c_proc_time_dark}{HTML}{6697EA}
\colorlet{c_proc_time}{c_green}
\definecolor{c_proc_time_light}{HTML}{86B7FF}
\colorlet{c_event_time}{c_blue}

\definecolor{c_watermark_dark}{HTML}{DD8D87}
\colorlet{c_watermark}{c_green}
\definecolor{c_watermark_light}{HTML}{FDADA7}
\definecolor{c_output_watermark}{HTML}{CCCCFF}

\colorlet{c_boundary}{c_proc_time}

\colorlet{c_frances}{YellowOrange}
\colorlet{c_tyler}{Orchid}

\definecolor{amethyst}{rgb}{0.9, 0.7, 1}

\colorlet{c_watermark_e}{c_watermark_light}
\colorlet{c_watermark_e_out}{c_watermark_dark}
\colorlet{c_watermark_f}{c_proc_time_light}
\colorlet{c_watermark_f_out}{c_proc_time_dark}
\colorlet{c_watermark_g}{c_yellow}
\colorlet{c_perfect_watermark}{c_green} %{c_watermark!66!black}

\tikzstyle{s_margin_op}=[s_timepoint, rotate=0, c_legend]
\tikzstyle{s_margin_data}=[s_margin_op, font=\itshape]

% Opacities
\def \ostate{.222}%.111}%.333} %.7070}
\ifdefined \oout
\else
	\def \oout{.444}%.333} %0.796875
\fi
\def \oretraction{.333} % 0.75
\def \otrigger{.5}

\def \ofadeA{.1} %.2
\def \ofadeB{.05} %.1
\def \ofadeC{.01} 

\ifdefined \BackBorder
\else
	\def \BackBorder{c_back_dark}
\fi

\ifdefined \BackBorderWidth
\else
	\def \BackBorderWidth{1ex}
\fi

\newcommand{\TikzDiagram}[2][8cm] {
\begin{tikzpicture}
[color=white,show background rectangle, background rectangle/.style={fill=\BackBorder},/tikz/inner frame sep=\BackBorderWidth,font=\sffamily]
	#2
\end{tikzpicture}
\newpage
}

\newcommand{\WindowBrace}[5] {
	\draw [line width=1pt,gray] (\ox + #1, #2) -- (#1 - #4, #2);
	\draw [line width=1pt,gray] (\ox + #1, #3) -- (#1 - #4, #3);
	\draw [line width=1pt,gray] (\ox + #1 - #4, #2) -- (#1 - #4, #3);
	\draw [line width=1pt,gray] (\ox + #1 - #4, {#2 - (#2 - #3) / 2}) -- (#1 - #4 - #4, {#2 - (#2 - #3) / 2});
	\node [gray,font=\large] at (\ox + #1 - #4 - #4 - #4, {#2 - (#2 - #3) / 2}) {#5};
}


\tikzstyle{s_ideal}=[ dotted,line width=1.5pt,color=c_ideal]
\tikzstyle{s_watermark}=[line width=1.5pt, solid, color=c_watermark]
\tikzstyle{s_output_watermark}=[line width=1.5pt, solid, color=c_output_watermark, dotted]
\tikzstyle{s_timepoint}=[font=\small]
\tikzstyle{s_timeline}=[solid,line width=1.5pt,color=c_timeline]
\tikzstyle{s_boundary}=[color=c_boundary, densely dashed]
\tikzstyle{s_late}=[color=c_state_back,pattern=north west lines,pattern color=c_state_back]
\tikzstyle{s_legend}=[font=\small,c_legend]
\tikzstyle{s_tomb}=[color=c_tomb,densely dotted, line width=.5pt]
\tikzstyle{s_tomb_late}=[color=c_tomb,densely dotted, line width=.5pt]
	%\tikzstyle{s_tomb_late}=[color=white,densely dotted,pattern=crosshatch dots,pattern color=c_tomb] 

\ifdefined \DrawDefaultLegend
\else
	\def \DrawDefaultLegend{1}
\fi

% Default origin to (0, 0)
\ifdefined \ox
\else
	\def \ox{0}
\fi
\ifdefined \oy
\else
	\def \oy{0}
\fi


% Default to input version #1
\ifdefined \InputVersion
\else
	\def \InputVersion{1}
\fi

% Default to showing late data
\ifdefined \ShowLateData
\else
	\def \ShowLateData{1}
\fi

\ifdefined \InputVersionSet
\else
	\def \InputVersionSet{1}
\fi

\newcommand{\Version}[9] {
	\if \InputVersionSet 1
	\if \InputVersion 1
		#1 % Standard v1
	\else
		\if \InputVersion 2
			#2 % Standard v2 colored
		\else
			\if \InputVersion 3
				#3 % Ingress Shift v1
			\else
				\if \InputVersion 4
					#4 % Ingress Shift v2
				\else
					\if \InputVersion 5
						#5 % SumA
					\else
						\if \InputVersion 6
							#6 % SumB
						\else
							\if \InputVersion 7
								#7 % Top
							\else
								\if \InputVersion 8
									#8 % Input toggle
								\else
									\if \InputVersion 9
										#9 % Percentile watermarks
									\fi
								\fi
							\fi
						\fi
					\fi
				\fi
			\fi
		\fi
	\fi
	\fi
}

\newcommand{\VersionMore}[5] {
	\if \InputVersionSet 2
		\if \InputVersion 1
			#1 % Re-window key #1
		\else
			\if \InputVersion 2
				#2 % Re-window key #2
			\else
				\if \InputVersion 3
					#3 % Re-window global
				\else
					\if \InputVersion 4
						#4 % Window-alignment key #1
					\else
						\if \InputVersion 5
							#5 % Re-window key #2
						\fi
					\fi
				\fi
			\fi
		\fi
	\fi
}

\newcommand{\ResetMaxx} {
\Version
  {\def \maxx{9}}
  {\def \maxx{9}}
  {\def \maxx{10}}
  {\def \maxx{10}}
  {\def \maxx{9}}
  {\def \maxx{9}}
  {\def \maxx{9}}
  {\def \maxx{9}}
  {\def \maxx{9}}
\VersionMore
  {\def \maxx{10}}
  {\def \maxx{10}}
  {\def \maxx{10}}
  {\def \maxx{10}}
  {\def \maxx{10}}
}
\ResetMaxx  

\ifdefined \maxy
\else
	\def \maxy{5}
\fi
\def \yoff{5}
\def \scale{.9}
\def \trioff{0.0265}

\newcommand {\UserScoreAxes}[0] {
	
	% X Axis
	\draw[color=c_event_time, line width=1.5pt] (\ox,\oy) -- (\ox + \maxx, \oy);
	% X Axis Tick Marks
	\foreach \x in {0,1,2,3,4,5,6,7,8,9,10,11,12,13,14,15,16,17,18,19,20,21,22,23} {
		\draw[color=c_event_time] (\ox + \x, \oy + .1) -- (\ox + \x, \oy - .1);
		\node[s_timepoint,color=c_event_time] at (\ox + \x, \oy - .25) {12:0\x};
	}
	% X Axis Label
	\node[s_legend,color=c_event_time] at (\ox + \maxx / 2, \oy - .666) {Event Time};

	% Y Axis
	\draw[c_proc_time, line width=1.5pt] (\ox, \oy - 0.0265) -- (\ox, \oy + \maxy);
	% Y Axis Tick Marks
	\foreach \y / \t in {1/06, 2/07, 3/08, 4/09, 5/10} {
		\draw[c_proc_time] (\ox + .1, \oy + \y) -- (\ox - .1, \oy + \y);
		\node[s_timepoint, rotate=90, c_proc_time] at (\ox - .25, \oy + \y) {12:\t};
	}
	% Y Axis Label
	\node[s_legend,rotate=90, c_proc_time] at (\ox -.666, \oy + \maxy / 2) {Processing Time};	
	
	% Corner beautification
	\fill[c_event_time, line width=0pt] (\ox - \trioff, \oy - \trioff) -- (\ox + \trioff, \oy + \trioff) -- (\ox + \trioff, \oy - \trioff) -- (\ox - \trioff, \oy - \trioff);
}

\newdimen \xdim
\newdimen \maxxdim


\newcommand {\DrawAxes}[0] {
	
	% X Axis
	\draw[color=c_event_time, line width=1.5pt] (\ox,\oy) -- (\ox + \maxx, \oy);
	% X Axis Tick Marks
	\if \XAxisToTen 0
		\foreach \x in {0,...,\maxx} {
			\draw[color=c_event_time] (\ox + \x, \oy + .1) -- (\ox + \x, \oy - .1);
			\node[s_timepoint,color=c_event_time] at (\ox + \x, \oy - .25) {12:0\x};
		}
	\else
		\foreach \x / \t in {0/00, 1/01, 2/02, 3/03, 4/04, 5/05, 6/06, 7/07, 8/08, 9/09, 10/10} {
			\draw[color=c_event_time] (\ox + \x, \oy + .1) -- (\ox + \x, \oy - .1);
			\node[s_timepoint,color=c_event_time] at (\ox + \x, \oy - .25) {12:\t};
		}
	\fi
	% X Axis Label
	\contourlength{1pt}
	\node[s_legend,c_event_time] at (\ox + \maxx / 2, \oy - .666) {Event Time};

	% Y Axis
	\draw[c_proc_time, line width=1.5pt] (\ox, \oy - 0.0265) -- (\ox, \oy + \maxy);
	% Y Axis Tick Marks
	\if \YAxisToEleven 0
		\foreach \y / \t in {1/06, 2/07, 3/08, 4/09, 5/10} {
			\draw[c_proc_time] (\ox + .1, \oy + \y) -- (\ox - .1, \oy + \y);
			\node[s_timepoint, rotate=90, c_proc_time] at (\ox - .25, \oy + \y) {12:\t};
		}
	\else
		\foreach \y / \t in {1/06, 2/07, 3/08, 4/09, 5/10, 6/11} {
			\draw[c_proc_time] (\ox + .1, \oy + \y) -- (\ox - .1, \oy + \y);
			\node[s_timepoint, rotate=90, c_proc_time] at (\ox - .25, \oy + \y) {12:\t};
		}
	\fi
	% Y Axis Label
	\node[s_legend,rotate=90, c_proc_time] at (\ox -.666, \oy + \maxy / 2) {Processing Time};	
	
	% Corner beautification
	\fill[c_event_time, line width=0pt] (\ox - \trioff, \oy - \trioff) -- (\ox + \trioff, \oy + \trioff) -- (\ox + \trioff, \oy - \trioff) -- (\ox - \trioff, \oy - \trioff);
}

\contourlength{1pt}

\newcommand {\DrawInputOld}[6] {
	\filldraw[#1, draw=#6] (\ox + #2, \oy + #3) circle [radius=6pt];
	\contourlength{\InputContour}
	\node[#5,font=\bf\sffamily] at (\ox + #2, \oy + #3) {\contour{#6}{#4}};
}

\newcommand{\DrawInputNode}[5] {
	\node[#4,font=\bf\sffamily,thick,#5,text opacity=0] at (\ox + #1, \oy + #2) {x};
	\node[#4,font=\bf\sffamily,thick,draw opacity=0,fill opacity=0,text opacity=1] at (\ox + #1, \oy + #2) {#3};
}

\newcommand{\DrawInputNodeWonky}[5] {
	\node[#4,font=\bf\sffamily,thick,#5] at ({\ox + #1}, \oy + #2) {#3};
}

\ifdefined \InputStyle
\else
	\def \InputStyle{0}
\fi

\newcommand {\DrawInput}[6][white] {
	\if \InputStyle 0
		\DrawInputOld{#2}{#3}{#4}{#5}{#6}{#1}
	\else
		\tikzstyle s_common=[fill=#2,draw=#6,text=#6]
		\def \InputX{#3}
		\def \InputY{#4}
		\if \InputStyle 1
			\DrawInputNodeWonky{\InputX}{\InputY}{#5}{s_common}{circle, inner sep=3pt}
		\else
			\if \InputStyle 2
				\DrawInputNode{\InputX}{\InputY}{#5}{s_common}{rectangle, inner sep=4.5pt}
			\else
				\if \InputStyle 3
					\DrawInputNode{\InputX}{\InputY}{#5}{s_common}{diamond, inner sep=2.5pt}
				\else
					\DrawInputNodeWonky{\InputX}{\InputY}{#5}{s_common}{rectangle, inner sep=4.5pt}
				\fi
			\fi
		\fi
	\fi
}

	% A inputs, in sorted order (by processing-time)
	%\DrawInput{0.444}{.333}{5}
	%\DrawInput{2.444}{.666}{7}
	%\DrawInput{3.666}{1.222}{3}
	%\DrawInput{3.888}{1.444}{4}
	%\DrawInput{4.333}{1.666}{3}
	%\DrawInput{3.111}{2.111}{8}
	%\DrawInput{6.666}{2.333}{3}
	%\DrawInput{1.444}{3.333}{9}
	%\DrawInput{7.444}{3.666}{8}
	%\DrawInput{7.777}{4.0}{1}

\ifdefined \InputsAFourBump
\else
	\def \InputsAFourBump{0}
\fi

\ifdefined \InputsAEightBump
\else
	\def \InputsAEightBump{0}
\fi

\newcommand {\DrawInputsA}[4][0] {
	% #1 = y offset
	% #2 = back color
	% #3 = text color
	% #4 = border color
	\DrawInput[#4]{#2}{0.444}{.333 + #1}{5}{#3}
	\DrawInput[#4]{#2}{2.444}{.666 + #1}{7}{#3}
	\DrawInput[#4]{#2}{3.111}{2.111 + #1 + \InputsAEightBump}{8}{#3}
	\DrawInput[#4]{#2}{3.666}{1.222 + #1}{3}{#3}
	%\DrawInput{#2}{3.888}{1.444 + #1}{4}{#3}
	\DrawInput[#4]{#2}{4.333}{1.666 + #1 + \InputsAFourBump}{4}{#3}. % used to be 3
	\DrawInput[#4]{#2}{6.666}{2.333 + #1}{3}{#3}
	\DrawInput[#4]{#2}{7.444}{3.666 + #1 - \InputsAEightBump}{8}{#3}
	\DrawInput[#4]{#2}{7.777}{4.0 + #1}{1}{#3}
	\if \ShowLateData 1
		\def \LateNineColor{#2}
		\def \LateNineBorder{#4}
		\if \IsGhost 0
			\if \ShowLateness 1
				\def \LateNineColor{s_dropped}
				\def \LateNineBorder{c_dropped_border}
			\else 
				\if \NineIsLate 1
					\def \LateNineColor{s_dropped}
					\def \LateNineBorder{c_dropped_border}
				\else
				\fi
			\fi
		\fi
		\DrawInput[\LateNineBorder]{\LateNineColor}{1.444}{3.333 + #1}{9}{#3}
	\fi
	\if \ShowLateness 1
		\DrawInput[#4]{#2}{.888}{1.666 + #1}{6}{#3}
	\fi	
}

	% B inputs, in sorted order (by processing-time)
	%\DrawInput{#1}{2.444}{.888}{7}
	%\DrawInput{#1}{3.666}{1.111}{3}
	%\DrawInput{#1}{1.444}{1.333}{9}
	%\DrawInput{#1}{3.111}{1.777}{8}
	%\DrawInput{#1}{6.666}{1.999}{3}
	%\DrawInput{#1}{0.444}{2.111}{5}
	%\DrawInput{#1}{7.444}{2.555}{8}
	%\DrawInput{#1}{3.888}{2.888}{4}
	%\DrawInput{#1}{7.777}{3.222}{1}
	%\DrawInput{#1}{4.333}{3.888}{3}

\newcommand {\DrawInputsB}[3] {
	% #1 = back color
	% #2 = text color
	% #3 = border color
	\DrawInput[#3]{#1}{0.444}{2.111}{5}{#2}
	\DrawInput[#3]{#1}{2.444}{.888}{7}{#2}
	\DrawInput[#3]{#1}{3.111}{1.777}{8}{#2}
	\DrawInput[#3]{#1}{3.666}{1.111}{3}{#2}
	%\DrawInput[#3]{#1}{3.888}{2.888}{4}{#2}
	\DrawInput[#3]{#1}{4.333}{3.888}{4}{#2} % used to be 3
	\DrawInput[#3]{#1}{6.666}{2.000}{3}{#2}
	\DrawInput[#3]{#1}{7.444}{2.555}{8}{#2}
	\DrawInput[#3]{#1}{7.777}{3.222}{1}{#2}
	\DrawInput[#3]{#1}{1.444}{1.333}{9}{#2}f
}

\tikzstyle{s_datum_ghost}=[color=c_datum_back_ghost,opacity=.333]

\newcommand{\DrawShiftedInput}[2] {
	\DrawInput{c_datum_back}{5 + #1}{#1}{#2}{c_datum_text}
}

\newdimen \deltaDim
\newdimen \offDim

\newcommand{\DrawIngressInput}[6] {
	\def \xL{#3}
	\def \xR{5 + #4}
	\def \arrowOff{.3}

	% Floating point calculations have to be done via
	% dimensions, which have to be set using evaluated
	% numbers, which appears to doable with \pgfmathparse
	% and \pgfmathresult...
	\pgfmathparse{\xR - \xL}
	\deltaDim=\pgfmathresult pt
	
	\pgfmathparse{\arrowOff * 2}
	\offDim=\pgfmathresult pt

	\contourlength{\InputContour}
	\DrawInput[c_ghost_border]{s_datum_ghost}{\xL}{#4}{#5}{c_datum_text}
	\if \IsGhost 0
		\ifthenelse{\lengthtest{\deltaDim > \offDim}}
			{\draw[s_datum_ghost,-stealth,densely dotted,line width=.75pt,draw=c_ghost_arrow] (\ox + \xL + \arrowOff, \oy + #4) -- (\ox + \xR - \arrowOff, \oy + #4);}
			{}
		\DrawInput[#6]{#1}{\xR}{#4}{#5}{#2}
	\fi % \IsGhost 0
}

\newcommand {\DrawInputsC}[3] {
	% #1 = back color
	% #2 = text color
	% #3 = border color
	\DrawIngressInput{#1}{#2}{0.444}{.333}{5}{#3}
	\DrawIngressInput{#1}{#2}{2.444}{.666}{7}{#3}
	\DrawIngressInput{#1}{#2}{3.666}{1.222}{3}{#3}
	%\DrawIngressInput{#1}{#2}{3.888}{1.444}{4}{#3}
	\DrawIngressInput{#1}{#2}{4.333}{1.666}{4}{#3} % used to be 3
	\DrawIngressInput{#1}{#2}{3.111}{2.111}{8}{#3}
	\DrawIngressInput{#1}{#2}{6.666}{2.333}{3}{#3}
	\DrawIngressInput{#1}{#2}{1.444}{3.333}{9}{#3}
	\DrawIngressInput{#1}{#2}{7.444}{3.666}{8}{#3}
	\DrawIngressInput{#1}{#2}{7.777}{4.0}{1}{#3}
}

\newcommand {\DrawInputsD}[3] {
	% #1 = back color
	% #2 = text color
	% #3 = border color
	\DrawIngressInput{#1}{#2}{2.444}{.888}{7}{#3}
	\DrawIngressInput{#1}{#2}{3.666}{1.111}{3}{#3}
	\DrawIngressInput{#1}{#2}{1.444}{1.333}{9}{#3}
	\DrawIngressInput{#1}{#2}{3.111}{1.777}{8}{#3}
	\DrawIngressInput{#1}{#2}{6.666}{2.000}{3}{#3}
	\DrawIngressInput{#1}{#2}{0.444}{2.111}{5}{#3}
	\DrawIngressInput{#1}{#2}{7.444}{2.555}{8}{#3}
	%\DrawIngressInput{#1}{#2}{3.888}{2.888}{4}{#3}
	\DrawIngressInput{#1}{#2}{7.777}{3.222}{1}{#3}
	\DrawIngressInput{#1}{#2}{4.333}{3.888}{4}{#3} % used to be 3
}

\newcommand{\DrawInputsE}[3] {
	\def \InputStyle{3}
	\DrawInputsA{#1}{#2}{#3}
}

\newcommand {\DrawInputsF}[2] {
	\def \InputStyle{2}
	\DrawInput{#1}{0.666}{0.777}{6}{#2}
	%\DrawInput{#1}{1.444}{0.333}{8}{#2}
	\DrawInput{#1}{2.666}{1.666}{1}{#2}
	%\DrawInput{#1}{5.111}{1.666}{5}{#2}
	\DrawInput{#1}{3.888}{2.888}{9}{#2}
	\DrawInput{#1}{5.888}{2.111}{3}{#2}
	\DrawInput{#1}{6.666}{2.000}{9}{#2}
	\DrawInput{#1}{6.222}{2.555}{2}{#2}
	\DrawInput{#1}{7.777}{3.222}{4}{#2}
	\DrawInput{#1}{1.222}{1.333}{7}{#2}
}

\newcommand {\DrawInputsG}[4] {
	\def \InputStyle{3}
	\DrawInput{#1}{1.444}{0.833}{1.0}{#3} % A
	\DrawInput{#1}{5.333}{2.600}{2.9}{#3} % A
	\DrawInput{#1}{8.777}{4.350}{2.1}{#3} % A
	%\DrawInput{#1}{6.666}{4.37}{12}{#3} % A
	\def \InputStyle{2}
	\DrawInput{#2}{2.222}{1.850}{1.6}{#4} % B
	\DrawInput{#2}{3.666}{3.150}{1.0}{#4} % B
	\DrawInput{#2}{4.888}{3.250}{1.0}{#4} % B
	\DrawInput{#2}{7.666}{3.550}{1.8}{#4} % B
	\DrawInput{#2}{8.777}{4.000}{1.0}{#4} % B
}

\newcommand {\DrawInputsH}[2] {
	%\DrawInput{#1}{0.333}{.222}{1}{#2}
	%\DrawInput{#1}{0.777}{.444}{1}{#2}
	%\DrawInput{#1}{1.555}{.555}{1}{#2}
	%\DrawInput{#1}{0.555}{.666}{1}{#2}
	%\DrawInput{#1}{3.333}{.888}{1}{#2}
	%\DrawInput{#1}{5.222}{.8888}{1}{#2}
	%\DrawInput{#1}{0.999}{.999}{1}{#2}
	%\DrawInput{#1}{2.888}{1.111}{1}{#2}
	%\DrawInput{#1}{4.888}{1.111}{1}{#2}
	%\DrawInput{#1}{2.333}{1.222}{1}{#2}
	%\DrawInput{#1}{1.777}{1.222}{1}{#2}
	%\DrawInput{#1}{0.444}{1.333}{1}{#2}
	%\DrawInput{#1}{5.444}{1.333}{1}{#2}
	%\DrawInput{#1}{3.666}{1.555}{1}{#2}
	%\DrawInput{#1}{4.777}{1.555}{1}{#2}
	%\DrawInput{#1}{1.222}{1.666}{1}{#2}
	%\DrawInput{#1}{5.777}{1.777}{1}{#2}
	%\DrawInput{#1}{6.555}{1.888}{1}{#2}
	%\DrawInput{#1}{0.888}{1.999}{1}{#2}
	%\DrawInput{#1}{1.888}{2.111}{1}{#2}
	%\DrawInput{#1}{4.222}{2.111}{1}{#2}
	%\DrawInput{#1}{6.111}{2.111}{1}{#2}
	%\DrawInput{#1}{3.111}{2.222}{1}{#2}
	%\DrawInput{#1}{4.888}{2.222}{1}{#2}
	%\DrawInput{#1}{4.111}{2.333}{1}{#2}
	%\DrawInput{#1}{6.333}{2.555}{1}{#2}
	%\DrawInput{#1}{5.888}{2.666}{1}{#2}
	%\DrawInput{#1}{4.555}{2.666}{1}{#2}
	%\DrawInput{#1}{2.222}{2.777}{1}{#2}
	%\DrawInput{#1}{3.888}{2.777}{1}{#2}
	%\DrawInput{#1}{7.111}{2.888}{1}{#2}
	%\DrawInput{#1}{3.555}{3.000}{1}{#2}
	%\DrawInput{#1}{2.888}{3.111}{1}{#2}
	%\DrawInput{#1}{6.888}{3.444}{1}{#2}
	%\DrawInput{#1}{3.444}{3.555}{1}{#2}
	%\DrawInput{#1}{7.888}{3.555}{1}{#2}
	%\DrawInput{#1}{7.333}{3.888}{1}{#2}
	%\DrawInput{#1}{7.555}{4.333}{1}{#2}
	%\DrawInput{#1}{7.222}{4.555}{1}{#2}
	%\DrawInput{#1}{7.888}{4.777}{1}{#2}


%----

	\DrawInput{#1}{0.333}{.222}{1}{#2}
	\DrawInput{#1}{0.555}{.666}{1}{#2}
	\DrawInput{#1}{0.777}{.444}{1}{#2}
	\DrawInput{#1}{1.555}{.555}{1}{#2}
	\DrawInput{#1}{0.999}{.999}{1}{#2}
	\DrawInput{#1}{0.444}{1.333}{1}{#2}
	\DrawInput{#1}{1.222}{1.666}{1}{#2}
	\DrawInput{#1}{1.888}{2.111}{1}{#2}
	\DrawInput{#1}{0.888}{1.999}{1}{#2}
	\DrawInput{#1}{1.777}{1.222}{1}{#2}

	\DrawInput{#1}{2.333}{1.222}{1}{#2}
	\DrawInput{#1}{2.888}{1.111}{1}{#2}
	\DrawInput{#1}{3.333}{.888}{1}{#2}
	\DrawInput{#1}{3.666}{1.555}{1}{#2}
	\DrawInput{#1}{3.111}{2.222}{1}{#2}
	\DrawInput{#1}{2.222}{2.777}{1}{#2}
	\DrawInput{#1}{2.888}{3.111}{1}{#2}
	\DrawInput{#1}{3.888}{2.777}{1}{#2}
	\DrawInput{#1}{3.444}{3.555}{1}{#2}
	\DrawInput{#1}{3.555}{3.000}{1}{#2}

	\DrawInput{#1}{4.111}{2.333}{1}{#2}
	\DrawInput{#1}{4.222}{2.111}{1}{#2}
	\DrawInput{#1}{4.555}{2.666}{1}{#2}
	\DrawInput{#1}{4.888}{2.222}{1}{#2}
	\DrawInput{#1}{4.777}{1.555}{1}{#2}
	\DrawInput{#1}{5.777}{1.777}{1}{#2}
	\DrawInput{#1}{5.888}{2.666}{1}{#2}
	\DrawInput{#1}{5.444}{1.333}{1}{#2}
	\DrawInput{#1}{4.888}{1.111}{1}{#2}
	\DrawInput{#1}{5.222}{.888}{1}{#2}


	\DrawInput{#1}{6.111}{2.111}{1}{#2}
	\DrawInput{#1}{6.555}{1.888}{1}{#2}
	\DrawInput{#1}{6.333}{2.555}{1}{#2}
	\DrawInput{#1}{6.888}{3.444}{1}{#2}
	\DrawInput{#1}{7.888}{3.555}{1}{#2}
	\DrawInput{#1}{7.333}{3.888}{1}{#2}
	\DrawInput{#1}{7.111}{2.888}{1}{#2}
	\DrawInput{#1}{7.555}{4.333}{1}{#2}
	\DrawInput{#1}{7.888}{4.777}{1}{#2}
	\DrawInput{#1}{7.222}{4.555}{1}{#2}

}

\newcommand {\DrawInputsI}[3] {
	%\def \InputStyle{3}
	\DrawInput[#3]{#1}{0.333}{.333}{5}{#2}
	\DrawInput[#3]{#1}{1.555}{.666}{9}{#2}
	\DrawInput[#3]{#1}{2.777}{.777}{7}{#2}
	\DrawInput[#3]{#1}{3.222}{2.111}{8}{#2}
	\DrawInput[#3]{#1}{3.999}{1.444}{4}{#2}
	\DrawInput[#3]{#1}{4.444}{1.666}{3}{#2}
	\DrawInput[#3]{#1}{4.777}{1.222}{3}{#2}
	\DrawInput[#3]{#1}{6.666}{2.333}{3}{#2}
	\DrawInput[#3]{#1}{7.444}{3.666}{8}{#2}
	\DrawInput[#3]{#1}{7.777}{4.0}{1}{#2}
}

\newcommand {\DrawInputsJ}[3] {
	%\def \InputStyle{2}
	\DrawInput[#3]{#1}{0.333}{0.333}{6}{#2}
	\DrawInput[#3]{#1}{1.222}{.666}{4}{#2}
	\DrawInput[#3]{#1}{1.999}{0.3}{3}{#2}
	\DrawInput[#3]{#1}{2.222}{.666}{4}{#2}
	\DrawInput[#3]{#1}{2.888}{2.111}{9}{#2}
	\DrawInput[#3]{#1}{3.95}{1.555}{8}{#2}
	\DrawInput[#3]{#1}{5.5}{2.222}{7}{#2}
	\DrawInput[#3]{#1}{5.888}{1.333}{1}{#2}
	\DrawInput[#3]{#1}{7.000}{2.333}{5}{#2}
	\DrawInput[#3]{#1}{7.444}{3.444}{2}{#2}
	\DrawInput[#3]{#1}{8.333}{4.0}{6}{#2}
}

\newcommand {\DrawInputsK}[4] {
	\def \InputStyle{3}
	\DrawInput{#1}{1.333}{0.833}{1.0}{#3}
	\DrawInput{#1}{2.555}{1.05}{1.0}{#3}
	\DrawInput{#1}{5.333}{2.666}{3.0}{#3}
	\DrawInput{#1}{8.777}{4.533}{2.1}{#3}

	\def \InputStyle{2}
	\DrawInput{#2}{3.888}{2.5}{3.5}{#4}
	\DrawInput{#2}{4.95}{2.6}{1.0}{#4}
	\DrawInput{#2}{6.888}{3.2}{1.4}{#4}
	\DrawInput{#2}{9.333}{4.833}{2.3}{#4}
}

\newcommand {\DrawInputsL}[3] {
	%\def \InputStyle{3}
	\DrawInput[#3]{#1}{0.333}{.333}{5}{#2}
	\DrawInput[#3]{#1}{1.555}{.666}{9}{#2}
	
	\DrawInput[#3]{#1}{2.777}{.777}{7}{#2}
	\DrawInput[#3]{#1}{3.222}{2.111}{8}{#2}
	
	\DrawInput[#3]{#1}{4.000}{1.444}{4}{#2}
	\DrawInput[#3]{#1}{4.444}{1.666}{3}{#2}
	\DrawInput[#3]{#1}{4.777}{1.222}{3}{#2}
	
	\DrawInput[#3]{#1}{6.666}{2.333}{3}{#2}
	\DrawInput[#3]{#1}{7.444}{3.666}{8}{#2}
	\DrawInput[#3]{#1}{7.777}{4.0}{1}{#2}

	\DrawInput[#3]{#1}{8.111}{4.111}{6}{#2}
	\DrawInput[#3]{#1}{9.15}{4.555}{5}{#2}
}

\newcommand {\DrawInputsM}[3] {
	%\def \InputStyle{2}
	\DrawInput[#3]{#1}{0.333}{0.333}{6}{#2}
	
	\DrawInput[#3]{#1}{1.222}{.666}{4}{#2}

	\DrawInput[#3]{#1}{2.000}{0.3}{3}{#2}
	\DrawInput[#3]{#1}{2.222}{.666}{4}{#2}
	\DrawInput[#3]{#1}{2.888}{2.111}{9}{#2}	

	\DrawInput[#3]{#1}{3.95}{1.555}{8}{#2}
	
	\DrawInput[#3]{#1}{5.5}{2.222}{7}{#2}
	\DrawInput[#3]{#1}{5.888}{1.333}{1}{#2}
	
	\DrawInput[#3]{#1}{7.000}{2.333}{5}{#2}
	\DrawInput[#3]{#1}{7.444}{3.444}{2}{#2}
	
	\DrawInput[#3]{#1}{8.333}{4.0}{6}{#2}
}

\newcommand {\DrawInputsN}[4] {
	\def \InputStyle{3}
	\DrawInput{#1}{0.444}{1.000}{99}{#3} % A
	\DrawInput{#1}{2.444}{2.5}{22}{#3} % A
	\DrawInput{#1}{4.333}{2.675}{3}{#3} % A
	\DrawInput{#1}{6.666}{4.37}{12}{#3} % A
	\def \InputStyle{2}
	\DrawInput{#2}{0.666}{1.8}{13}{#4} % B
	\DrawInput{#2}{2.666}{3.255}{10}{#4} % B
	\DrawInput{#2}{5.888}{3.333}{3}{#4} % B
	\DrawInput{#2}{6.222}{3.61}{15}{#4} % B
}


\newcommand{\DrawGhostInputs} {
	\def \IsGhost{1}
	\Version
	  {\DrawInputsA{c_ghost}{c_ghost_text}{c_ghost_border}}
	  {\DrawInputsB{c_ghost}{c_ghost_text}{c_ghost_border}}
	  {\DrawInputsC{c_ghost}{c_ghost_text}{c_ghost_border}}
	  {\DrawInputsD{c_ghost}{c_ghost_text}{c_ghost_border}}
	  {\DrawInputsE{c_ghost}{c_ghost_text}{g_ghost_border}}
	  {\DrawInputsF{c_ghost}{c_ghost_text}}
	  {\DrawInputsG{c_ghost}{c_ghost}{c_ghost_text}{c_ghost_text}}
	  {}
	  {\DrawInputsH{c_ghost}{c_ghost_text}}
	\VersionMore
	  {\DrawInputsI{c_ghost}{c_ghost_text}{c_ghost}}
	  {\DrawInputsJ{c_ghost}{c_ghost_text}{c_ghost}}
	  {\DrawInputsK{c_ghost}{c_ghost}{c_ghost_text}{c_ghost_text}}
	  {\DrawInputsL{c_ghost}{c_ghost_text}{c_ghost_border}}
	  {\DrawInputsM{c_ghost}{c_ghost_text}{c_ghost_border}}
	\def \IsGhost{0}
}
	
\newcommand {\DrawInputs} {
	\Version
	  {\DrawInputsA{c_datum_back}{c_datum_text}{c_datum_border}}
	 % {\DrawInputsB{c_datum_text}{c_toggle_b}}
	  {\DrawInputsB{c_toggle_b}{c_datum_text}{c_datum_border}}
	  {\DrawInputsC{c_datum_back}{c_datum_text}{c_datum_border}}
	  {\DrawInputsD{c_toggle_b}{c_datum_text}{c_datum_border}}
	  {\DrawInputsE{c_watermark_e}{c_datum_text}{c_datum_border}}
	  {\DrawInputsF{c_watermark_f}{c_datum_text}}
	  {\DrawInputsG{c_datum_text}{c_datum_text}{c_watermark_e}{c_watermark_f}}
	  {}
	  {\DrawInputsH{c_datum_back}{c_datum_text}}
	\VersionMore
	  {\DrawInputsI{c_frances}{c_datum_text}{c_frances}}
	  {\DrawInputsJ{c_tyler}{c_datum_text}{c_tyler}}
	  {\DrawInputsK{c_frances}{c_tyler}{c_datum_text}{c_datum_text}}
	  {\DrawInputsL{c_datum_back}{c_datum_text}{c_datum_border}}
	  {\DrawInputsM{c_datum_back}{c_datum_text}{c_datum_border}}
}

\newcommand {\DrawMicroBatchBoundaries}[0] {
	\DrawBatchBoundary{1}
	%\DrawBatchBoundary{2}
	\DrawBatchBoundary{3}
	%\DrawBatchBoundary{4}
	\DrawBatchBoundary{5}
}

\newcommand{\DrawBatchBoundary}[1] {
	\draw[s_boundary] (\ox + 0, \oy + #1) -- (\ox + \maxx, \oy + #1);
}

\def \ShowJoints{1}

\newcommand{\DrawPerfectWatermarkA}[1] {
		\draw [-, s_watermark, #1] 
			(\ox + 0.111, \oy + 0.01)
			to [out=80,in=210] (\ox + 0.333, \oy +  .666) % 5
			to [out=30, in=180] (\ox + 1.666, \oy +  3.7) % 7
			to [out=0, in=180] (\ox + 7.7, \oy +  4.333) % 1
			to [out=0, in=180] (\ox + 9.000, \oy +  4.37) % end
		;
}

\newcommand{\DrawHeuristicWatermarkA}[1] {
		\draw [-, s_watermark, #1] 
			(\ox + 0.111, \oy + 0.01)
			to [out=80,in=180] (\ox + 0.444, \oy +  .666) % 5. 
			to [out=0, in=180] (\ox + 2.444, \oy +  .999) % 7
			to [out=0, in=180] (\ox + 3.111, \oy +  2.444) % 8
			to [out=0, in=180] (\ox + 6.555, \oy +  2.666) % 3 (\ox + rightmost)
			to [out=0, in=180] (\ox + 7.333, \oy +  4.000) % 8
			to [out=0, in=180] (\ox + 7.7, \oy +  4.333) % 1
			to [out=0, in=180] (\ox + 9.000, \oy +  4.37) % end
		;
}

\newcommand{\DrawHeuristicWatermarkAA}[1] {
		\draw [-, s_watermark, #1] 
			(\ox + -0.111, \oy + 0.01)
			to [out=80,in=180] (\ox + 0.444, \oy +  .666) % 5
			to [out=0, in=180] (\ox + 2.444, \oy +  .999) % 7
			to [out=0, in=180] (\ox + 3.111, \oy +  2.444) % 8
			to [out=0, in=180] (\ox + 6.555, \oy +  2.666) % 3 (\ox + rightmost)
			to [out=0, in=180] (\ox + 7.333, \oy +  4.000) % 8
			to [out=0, in=180] (\ox + 7.7, \oy +  4.333) % 1
			to [out=0, in=180] (\ox + 9.000, \oy +  4.37) % end
		;
}


\ifdefined \UsePerfectWatermark
\else
	\def \UsePerfectWatermark{0}
\fi

\ifdefined \UseHeuristicWatermark
\else
	\def \UseHeuristicWatermark{1}
\fi

\ifdefined \DottedPerfectWatermark
\else
	\def \DottedPerfectWatermark{0}
\fi

\ifdefined \DottedHeuristicWatermark
\else
	\def \DottedHeuristicWatermark{0}
\fi

\ifdefined \FlushLeftWatermark
\else
	\def \FlushLeftWatermark{0}
\fi

\newcommand {\DrawWatermarkA}[1][c_watermark] {
	\if \UsePerfectWatermark 1
		\if \DottedPerfectWatermark 0
			\DrawPerfectWatermarkA{c_perfect_watermark, dashed}
		\fi
	\fi

	\if \UseHeuristicWatermark 1
		\if \DottedHeuristicWatermark 0
			\if \FlushLeftWatermark 1
				\DrawHeuristicWatermarkAA{#1}
			\else
				\DrawHeuristicWatermarkA{#1}
			\fi
		\fi
	\fi
	
	\if \UsePerfectWatermark 1
		\if \DottedPerfectWatermark 1
			\DrawPerfectWatermarkA{c_perfect_watermark, dotted}
		\fi
	\fi

	\if \UseHeuristicWatermark 1
		\if \DottedHeuristicWatermark 1
			\if \FlushLeftWatermark 1
				\DrawHeuristicWatermarkAA{c_watermark, dotted}
			\else
				\DrawHeuristicWatermarkA{c_watermark, dotted}
			\fi
		\fi
	\fi
}

\newcommand {\DrawWatermarkB}[1][c_watermark] {
	\draw [-, s_watermark, #1] 
		(\ox + 0.111, \oy + 0.01)
		to [out=80,in=180] (\ox + 1.444, \oy +  1.666) % 9
		to [out=0, in=180] (\ox + 3.111, \oy +  2.111) % 8
		to [out=0, in=180] (\ox + 3.888, \oy +  3.222) % 4
		to [out=0, in=180] (\ox + 7.333, \oy +  3.333) % 8
		to [out=0, in=180] (\ox + 7.666, \oy +  3.555) % 1
		to [out=0, in=225] (\ox + 10.000, \oy +  5.4) % end
	;
}

\newcommand{\DrawIdealWatermark}[1][s_ideal] {
	\draw [-,#1] (\ox + \yoff, \oy + 0) -- (\ox + \maxx, \oy +  \maxx - \yoff);
}

\newcommand{\WatermarkJoint}[4][.55pt]{
	\filldraw[draw=#2,fill=#2] (\ox + #3, \oy + #4) circle [radius=#1];
}

\newcommand{\DrawOutputWatermarkE}[2] {
	\draw [-, #1, #2] 
		(\ox + 0.111, \oy + 0.01)
		to [out=80,in=180] (\ox + 0.444, \oy +  0.666) % 5
		to [out=0, in=180] (\ox + 1.000,  \oy +  0.700) % 7
	;
	\draw [#1, #2] (\ox + 1.000, \oy + 0.700) -- (\ox + 1.000, \oy + 1.000);
	\draw [#1, #2] (\ox + 1.000, \oy + 0.963) -- (\ox + 2.000, \oy + 0.963);
	%\draw [-, #1, #2] 
	%	(\ox + 2.000, \oy + 0.963)
	%	to [out=10,in=180] (\ox + 2.444, \oy +  1.00) % 7
	%;
	\draw [#1, #2] (\ox + 2.000, \oy + 0.963) -- (\ox + 2.000, \oy + 1.505);
	\draw [#1, #2] (\ox + 2.000, \oy + 1.505) -- (\ox + 2.666, \oy + 1.505);
	\draw [#1, #2] (\ox + 2.666, \oy + 1.505) -- (\ox + 2.666, \oy + 2.505);
	\draw [#1, #2] (\ox + 2.666, \oy + 2.478) -- (\ox + 4.000, \oy + 2.478);
	\draw [-, #1, #2] 
		(\ox + 4.000, \oy + 2.478)
		to [out=4,in=188] (\ox + 4.333, \oy +  2.505) % 3
	;
	\draw [#1, #2] (\ox + 4.333, \oy + 2.505) -- (\ox + 4.333, \oy + 2.675);
	\draw [#1, #2] (\ox + 4.333, \oy + 2.655) -- (\ox + 5.333, \oy + 2.655);
	\draw [#1, #2] (\ox + 5.333, \oy + 2.655) -- (\ox + 5.333, \oy + 3.111);
	\draw [#1, #2] (\ox + 5.333, \oy + 3.111) -- (\ox + 6.777, \oy + 3.111);
	\draw [#1, #2] (\ox + 6.777, \oy + 3.111) -- (\ox + 6.777, \oy + 4.337);
	\draw [#1, #2] (\ox + 6.777, \oy + 4.337) -- (\ox + 8.000, \oy + 4.337);
	%\draw [-, #1, #2] 
	%	(\ox + 6.000, \oy + 2.655)
	%	to [out=0,in=180] (\ox + 6.333, \oy +  2.663) % 3 #2
	%	to [out=4,in=195] (\ox + 6.666, \oy +  2.685) % 3 #2
	%;
	%\draw [#1, #2] (\ox + 6.666, \oy + 2.685) -- (\ox + 6.666, \oy + 4.37);
	%\draw [#1, #2] (\ox + 6.666, \oy + 4.337) -- (\ox + 8.000, \oy + 4.337);
	\draw [-, #1, #2] 
		(\ox + 8.000, \oy + 4.337)
		to [out=2,in=180] (\ox + 9.000, \oy + 4.37)
	;

	\if \ShowJoints 1
		\WatermarkJoint{#2}{0.444}{0.666}
		\WatermarkJoint{#2}{2.000}{0.963}
		\WatermarkJoint{#2}{2.444}{1.000}
		\WatermarkJoint{#2}{4.000}{2.478}
		\WatermarkJoint{#2}{4.333}{2.505}
		\WatermarkJoint{#2}{6.666}{2.685}
	\fi
}

\newcommand{\DrawOutputWatermarkF}[2] {
	\draw [-, #1, #2] 
		(\ox + 0.111, \oy + 0.01)
		%to [out=80,in=231] (\ox + 0.666, \oy +  1.333) % 6
		to [out=80,in=180] (\ox + 1.444, \oy +  1.666) % 9
	;
	\draw [#1, #2] (\ox + 1.444, \oy + 1.666) -- (\ox + 1.444, \oy + 1.850);
	\draw [#1, #2] (\ox + 1.444, \oy + 1.850) -- (\ox + 2.000, \oy + 1.850);	
	\draw [-, #1, #2] 
		(\ox + 2.000, \oy + 1.850)
		%to [out=22, in=198] (\ox + 2.666, \oy +  2.043) % 1
		to [out=0, in=180] (\ox + 3.111, \oy +  2.111) % 8
	;	
	\draw [#1, #2] (\ox + 3.111, \oy + 2.111) -- (\ox + 3.111, \oy + 3.253);	
	\draw [#1, #2] (\ox + 3.111, \oy + 3.223) -- (\ox + 4.000, \oy + 3.223);	
	\draw [-, #1, #2] 
		(\ox + 4.000, \oy + 3.223)
		to [out=0, in=183] (\ox + 5.888, \oy +  3.293) % 3
	;	
	\draw [#1, #2] (\ox + 5.888, \oy + 3.293) -- (\ox + 5.888, \oy + 3.333);	
	\draw [#1, #2] (\ox + 5.888, \oy + 3.3) -- (\ox + 6.000, \oy + 3.3);	
	\draw [-, #1, #2] 
		(\ox + 6.000, \oy + 3.3)
		to [out=0, in=183] (\ox + 6.222, \oy +  3.308) % 3
	;	
	\draw [#1, #2] (\ox + 6.222, \oy + 3.3) -- (\ox + 6.222, \oy + 3.61);	
	\draw [#1, #2] (\ox + 6.222, \oy + 3.585) -- (\ox + 8.000, \oy + 3.585);	
	\draw [-, #1, #2] 
		(\ox + 8.000, \oy + 3.585)
		to [out=11, in=180] (\ox + 9.000, \oy +  3.888) % end
	;	

	\if \ShowJoints 1
		\WatermarkJoint{#2}{0.666}{1.333}
		\WatermarkJoint{#2}{0.666}{1.77}
		\WatermarkJoint{#2}{2.000}{1.77}
		\WatermarkJoint{#2}{2.666}{2.043}
		\WatermarkJoint{#2}{5.888}{3.293}
		\WatermarkJoint{#2}{6.000}{3.3}
		\WatermarkJoint{#2}{6.222}{3.308}
		\WatermarkJoint{#2}{6.222}{3.585}
		\WatermarkJoint{#2}{8.000}{3.585}
	\fi
}


\newcommand {\DrawWatermarkE}[2] {
	\DrawWatermarkA[#1]
	\DrawOutputWatermarkE{s_output_watermark}{#2}
}

\newcommand {\DrawWatermarkF}[2] {
	\DrawWatermarkB[#1]
	\DrawOutputWatermarkF{s_output_watermark}{#2}
}


\newcommand{\DrawInputWatermarkG}[2] {
	\draw [-, #1, #2] 
		(\ox + 0.111, \oy + 0.01)
		to [out=80,in=240] (\ox + 0.238, \oy +  0.555)
	;
	\draw [-, #1, #2] 
		(\ox + 0.238, \oy + 0.555)
		to [out=72,in=231] (\ox + 0.666, \oy +  1.333) % 6
	;
	\draw [#1, #2] (\ox + 0.666, \oy + 1.333) -- (\ox + 0.666, \oy + 1.77);
	\draw [#1, #2] (\ox + 0.666, \oy + 1.775) -- (\ox + 2.000, \oy + 1.77);	
	\draw [-, #1, #2] 
		(\ox + 2.000, \oy + 1.77)
		to [out=22, in=203] (\ox + 2.444, \oy +  1.963) % 1
	;	
	\draw [#1, #2] (\ox + 2.444, \oy + 1.963) -- (\ox + 2.444, \oy + 2.478);
	\draw [#1, #2] (\ox + 2.444, \oy + 2.478) -- (\ox + 2.666, \oy + 2.478);
	\draw [#1, #2] (\ox + 2.666, \oy + 2.478) -- (\ox + 2.666, \oy + 3.223);	
	\draw [#1, #2] (\ox + 2.666, \oy + 3.223) -- (\ox + 4.000, \oy + 3.223);	
	\draw [-, #1, #2] 
		(\ox + 4.000, \oy + 3.223)
		to [out=0, in=183] (\ox + 5.888, \oy +  3.293) % 3
	;	
	\draw [#1, #2] (\ox + 5.888, \oy + 3.293) -- (\ox + 5.888, \oy + 3.333);	
	\draw [#1, #2] (\ox + 5.888, \oy + 3.3) -- (\ox + 6.000, \oy + 3.3);	
	\draw [-, #1, #2] 
		(\ox + 6.000, \oy + 3.3)
		to [out=0, in=183] (\ox + 6.222, \oy +  3.308) % 3
	;	
	\draw [#1, #2] (\ox + 6.222, \oy + 3.3) -- (\ox + 6.222, \oy + 3.585);	
	\draw [#1, #2] (\ox + 6.222, \oy + 3.585) -- (\ox + 6.666, \oy + 3.585);	

	\draw [#1, #2] (\ox + 6.666, \oy + 3.585) -- (\ox + 6.666, \oy + 4.337);
	\draw [#1, #2] (\ox + 6.666, \oy + 4.337) -- (\ox + 8.000, \oy + 4.337);
	\draw [-, #1, #2] 
		(\ox + 8.000, \oy + 4.337)
		to [out=2,in=180] (\ox + 9.000, \oy + 4.37)
	;

	\if \ShowJoints 1
		\WatermarkJoint{#2}{0.238}{0.555}
		\WatermarkJoint{#2}{0.666}{1.333}
		\WatermarkJoint{#2}{0.666}{1.77}
		\WatermarkJoint{#2}{2.000}{1.77}
		\WatermarkJoint{#2}{2.444}{1.963}
		\WatermarkJoint{#2}{2.444}{2.478}
		\WatermarkJoint{#2}{2.666}{2.478}
		\WatermarkJoint{#2}{2.666}{3.223}
		\WatermarkJoint{#2}{4.000}{3.223}
		\WatermarkJoint{#2}{5.888}{3.293}
		\WatermarkJoint{#2}{6.222}{3.585}
		\WatermarkJoint{#2}{6.666}{3.585}
		\WatermarkJoint{#2}{6.666}{4.337}
		\WatermarkJoint{#2}{8.000}{4.337}
	\fi
}

\newcommand{\DrawOutputWatermarkG}[2] {
	\draw [-, #1, #2] 
		(\ox + 0.111, \oy + 0.01)
		to [out=80,in=240] (\ox + 0.238, \oy +  0.555)
	;
	\draw [-, #1, #2] 
		(\ox + 0.238, \oy + 0.555)
		%to [out=72,in=231] (\ox + 0.444, \oy +  0.996) % 6
		to [out=80,in=180] (\ox + 1.444, \oy +  1.666) % 9
	;
	\draw [#1, #2] (\ox + 1.444, \oy + 1.666) -- (\ox + 1.444, \oy + 1.850);
	\draw [#1, #2] (\ox + 1.444, \oy + 1.850) -- (\ox + 2.000, \oy + 1.850);	
%	\draw [#1, #2] (\ox + 0.444, \oy + 0.996) -- (\ox + 0.444, \oy + 1.76);
%	\draw [#1, #2] (\ox + 0.444, \oy + 1.76) -- (\ox + 2.000, \oy + 1.76);	
	%\WatermarkJoint[1.075pt]{#2}{2.000}{1.76}
	\draw [-, #1, #2] 
		(\ox + 2.000, \oy + 1.76)
		to [out=22, in=203] (\ox + 2.444, \oy +  1.963) % 1
	;	
	%\WatermarkJoint[1.075pt]{#2}{2.444}{1.963}
	\draw [#1, #2] (\ox + 2.666, \oy + 2.000) -- (\ox + 2.666, \oy + 2.505);
	\draw [#1, #2] (\ox + 2.666, \oy + 2.505) -- (\ox + 3.111, \oy + 2.505);	
	\draw [#1, #2] (\ox + 3.111, \oy + 2.505) -- (\ox + 3.111, \oy + 3.223);	
	\draw [#1, #2] (\ox + 3.111, \oy + 3.223) -- (\ox + 4.000, \oy + 3.223);	
	\draw [-, #1, #2] 
		(\ox + 4.000, \oy + 3.223)
		to [out=0, in=183] (\ox + 4.333, \oy +  3.225) % 3
	;	
	%\WatermarkJoint[1.075pt]{#2}{4.333}{3.225}
	\draw [#1, #2] (\ox + 4.333, \oy + 3.225) -- (\ox + 4.333, \oy + 3.333);	
	\draw [#1, #2] (\ox + 4.333, \oy + 3.3) -- (\ox + 6.000, \oy + 3.3);	
	\draw [-, #1, #2] 
		(\ox + 6.000, \oy + 3.3)
		to [out=0, in=183] (\ox + 6.222, \oy +  3.308) % 3
	;	
	%\WatermarkJoint[1.075pt]{#2}{6.222}{3.308}
	\draw [#1, #2] (\ox + 6.222, \oy + 3.308) -- (\ox + 6.222, \oy + 3.610);	
	\draw [#1, #2] (\ox + 6.222, \oy + 3.610) -- (\ox + 6.777, \oy + 3.610);	
	\draw [#1, #2] (\ox + 6.777, \oy + 3.610) -- (\ox + 6.777, \oy + 4.337);	
	\draw [#1, #2] (\ox + 6.777, \oy + 4.337) -- (\ox + 8.000, \oy + 4.337);
	\draw [-, #1, #2] 
		(\ox + 8.000, \oy + 4.337)
		to [out=2,in=180] (\ox + 9.000, \oy + 4.37)
	;

	%\if \ShowJoints 1
	%	\WatermarkJoint{#2}{0.238}{0.555}
	%	\WatermarkJoint{#2}{0.666}{1.333}
	%	\WatermarkJoint{#2}{0.666}{1.77}
	%	\WatermarkJoint{#2}{2.000}{1.77}
	%	\WatermarkJoint{#2}{2.444}{1.963}
	%	\WatermarkJoint{#2}{2.444}{2.478}
	%	\WatermarkJoint{#2}{2.666}{2.478}
	%	\WatermarkJoint{#2}{2.666}{3.223}
	%	\WatermarkJoint{#2}{4.000}{3.223}
	%	\WatermarkJoint{#2}{5.888}{3.293}
	%	\WatermarkJoint{#2}{6.222}{3.585}
	%	\WatermarkJoint{#2}{6.666}{3.585}
	%	\WatermarkJoint{#2}{6.666}{4.337}
	%	\WatermarkJoint{#2}{8.000}{4.337}
	%\fi
}

\newcommand {\DrawWatermarkG} {
	%\DrawInputWatermarkG{s_watermark,draw opacity=1,line width=2.5pt}{white}
	\DrawOutputWatermarkG{s_watermark,draw opacity=1,line width=2.5pt}{c_yellow}
	\DrawOutputWatermarkE{s_output_watermark,draw opacity=1, line width=1.5pt}{c_watermark_e_out}
	\DrawOutputWatermarkF{s_output_watermark,draw opacity=1, line width=1.5pt}{c_watermark_f_out}
}

\newcommand {\DrawWatermarkH}[1][c_watermark] {
	\draw [-, s_watermark, #1, dotted] 
		(\ox + 0.111, \oy + 0.01)
		to [out=80,in=180] (\ox + 1.444, \oy +  .777) 
		to [out=0, in=210] (\ox + 3.111, \oy +  1.333) 
		to [out=30, in=180] (\ox + 5.444, \oy +  1.888)
		to [out=0, in=210] (\ox + 7.333, \oy +  3.222)
		to [out=30, in=190] (\ox + 8, \oy +  3.375)
		to [out=10, in=210] (\ox + 9.333, \oy +  4.4)
	;

	\draw [-, s_watermark, #1, densely dotted] 
		(\ox + 0.111, \oy + 0.01)
		to [out=80,in=180] (\ox + 1.444, \oy +  1.444) 
		to [out=0, in=210] (\ox + 3.111, \oy +  2.666) 
		to [out=30, in=180] (\ox + 5.444, \oy +  3.111)
		to [out=0, in=210] (\ox + 7.333, \oy +  3.999)
		to [out=30, in=180] (\ox + 8, \oy +  4.2)
		to [out=0, in=210] (\ox + 9.333, \oy +  4.8)
	;

	\draw [-, s_watermark, #1] 
		(\ox + 0.111, \oy + 0.01)
		to [out=80,in=180] (\ox + 1.111, \oy +  2.333) 
		to [out=0, in=210] (\ox + 2.666, \oy +  3.555) 
		to [out=30, in=190] (\ox + 5.444, \oy +  4.111)
		to [out=10, in=210] (\ox + 7.333, \oy +  4.777)
		to [out=30, in=190] (\ox + 8, \oy +  5.05)
		to [out=10, in=200] (\ox + 9.111, \oy +  5.3)
	;
}

\newcommand{\DrawWatermarkIJK}[1] {
		\draw [-, s_watermark, #1] 
			(\ox + 0.111, \oy + 0.01)
			to [out=80,in=180] (\ox + 0.444, \oy +  .666) % 5
			to [out=0, in=180] (\ox + 2.444, \oy +  .999) % 7
			to [out=0, in=180] (\ox + 3.111, \oy +  2.444) % 8
			to [out=0, in=180] (\ox + 6.555, \oy +  2.666) % 3 (\ox + rightmost)
			to [out=0, in=180] (\ox + 7.333, \oy +  4.000) % 8
			to [out=0, in=180] (\ox + 7.7, \oy +  4.333) % 1
			to [out=0, in=205] (\ox + 9.000, \oy +  4.6) % end
			to [out=25, in=220] (\ox + 10.000, \oy +  5.4) % end
		;
}

\newcommand{\DrawWatermarkLM}[1] {
		\draw [-, s_watermark, #1] 
			(\ox - 0.333, \oy + 0.01)
			to [out=80,in=220] (\ox + 0.111, \oy +  .555) % 5
			to [out=40,in=205] (\ox + 0.444, \oy +  .777) % 5
			to [out=25,in=205] (\ox + 1.000, \oy +  1.000) % 5
			to [out=25, in=220] (\ox + 2.444, \oy +  2.222) % 7
			to [out=40, in=200] (\ox + 3.111, \oy +  2.555) % 8
			to [out=20, in=205] (\ox + 6.555, \oy +  3.666) % 3 (\ox + rightmost)
			to [out=25, in=220] (\ox + 7.333, \oy +  4.222) % 8
			to [out=40, in=205] (\ox + 7.7, \oy +  4.444) % 1 
			to [out=25, in=205] (\ox + 9.000, \oy +  4.8) % end
			to [out=25, in=220] (\ox + 10.000, \oy +  5.4) % end
		;
}

\newcommand{\DrawWatermark}[1][c_watermark] {
 	\Version
	  {\DrawWatermarkA[#1]}
	  {\DrawWatermarkB[#1]}
	  {\DrawIdealWatermark[s_watermark,c_perfect_watermark,dashed]}
	  {\DrawIdealWatermark[s_watermark,c_perfect_watermark,dashed]}
	  {\DrawWatermarkE{c_watermark_e}{c_watermark_e_out}}
	  {\DrawWatermarkF{c_watermark_f}{c_watermark_f_out}}
	  {\DrawWatermarkG}
	  {}
	  {\DrawWatermarkH[#1]}
	\VersionMore
	  {\DrawWatermarkIJK{#1}}
	  {\DrawWatermarkIJK{#1}}
	  {\DrawWatermarkIJK{#1}}
	  {\DrawWatermarkLM{#1}}
	  {\DrawWatermarkLM{#1}}
}

\newcommand {\DrawBatchWatermarkAtTo}[3][c_watermark] {
	\draw [->, s_watermark, #1] (\ox + 0, \oy +  #2) -- (\ox + #3, \oy +  #2 + #3 / 8 * .125);
}

\newcommand{\DrawTerminatedWatermark}[3][c_watermark] {
	\DrawBatchWatermarkAtTo[#1]{#2}{#3};
	\draw [#1,s_watermark, ->] (\ox + #3, \oy +  #2 + #3 / 8 * .125) -- (\ox + 8, \oy +  #2 + #3 / 8 * .125);
}

\newcommand {\DrawBatchWatermarkAt}[2][c_watermark] {
	\DrawBatchWatermarkAtTo[#1]{#2}{8}
}

\newcommand {\DrawGlobalBatchWatermark}[1][c_watermark] {
	\DrawBatchWatermarkAt[#1]{4.25}
}

\newcommand {\DrawMicroBatchWatermarks}[1][c_watermark] {
	\DrawBatchWatermarkAt[#1]{4}
	\DrawBatchWatermarkAt[#1]{3}
	\DrawTerminatedWatermark[#1,-]{2}{7}
	\DrawTerminatedWatermark[#1,-]{1}{6}
}

\newcommand{\WatermarkLegend}[5][0] {
    % #1 = extra offset
    % #2 = x
    % #3 = y
    % #4 = text
    % #5 = style 
    
    \node[s_legend] at (#2, #3) {#4:};
    \draw [#5, -] (#2 + 2 + #1, #3) -- (#2 + 4 + #1, #3);  
    %\node[s_legend] at (\ox + 2.5, \oy +  -1.55) {Ideal watermark:};
    %\draw [s_ideal, -] (\ox + 4.5, \oy +  -1.55) -- (\ox + 6.5, \oy +  -1.55);
}

\newcommand{\DrawIdealWatermarkLegend}[1] {
		\node[s_legend] at (\ox + 2.5, \oy +  #1) {Ideal watermark:};
		\draw [s_ideal, -] (\ox + 4.5, \oy +  #1) -- (\ox + 6.5, \oy +  #1);
}

\newcommand{\DrawHeuristicWatermarkLegend}[2] {
		\node[s_legend] at (\ox + 2.5, \oy +  #1) {Heuristic watermark:};
		\draw [s_watermark, -, #2] (\ox + 4.5, \oy + #1) -- (\ox + 6.5, \oy + #1);
}

\newcommand{\DrawPerfectWatermarkLegend}[2] {
		\node[s_legend] at (\ox + 2.5, \oy +  #1) {Perfect watermark:};
		\draw [s_watermark, -, c_perfect_watermark, #2] (\ox + 4.5, \oy + #1) -- (\ox + 6.5, \oy + #1);
}

\newcounter{LegendIndex}

\newcommand{\DrawWatermarks}[2][\DrawWatermark] {
	\begin{scope}
	% Blocker
	\clip (\ox + 0.01, \oy) rectangle (\ox + \maxx, \oy +  #2);

	% Ideal watermark line
	\DrawIdealWatermark;
	
	#1
	
	\end{scope}

	% Watermark s_legends
	\if \DrawDefaultLegend 1
		\setcounter{LegendIndex}{0}

		\def \wy{-1.15}

		\if \UsePerfectWatermark 1
			\if \DottedPerfectWatermark 1
				\DrawPerfectWatermarkLegend{\wy - \value{LegendIndex} * .4}{dotted}
			\else
				\DrawPerfectWatermarkLegend{\wy - \value{LegendIndex} * .4}{dashed}
			\fi
			\stepcounter{LegendIndex}
		\fi		
		
		\if \UseHeuristicWatermark 1
			\if \DottedHeuristicWatermark 1
				\DrawHeuristicWatermarkLegend{\wy - \value{LegendIndex} * .4}{dotted}
			\else
				\DrawHeuristicWatermarkLegend{\wy - \value{LegendIndex} * .4}{}
			\fi
			\stepcounter{LegendIndex}
		\fi

		\DrawIdealWatermarkLegend{\wy - \value{LegendIndex} * .4}
	\fi

}

\contourlength{1pt}

\def \DrawTriggerLine{1}

\ifdefined \DrawTriggerLineBottom
\else
	\def \DrawTriggerLineBottom{0}
\fi

\newcommand {\DrawCustomOutput}[9][] {
	% arguments: 
	% #1 = extra box formatting
	% #2 = x1
	% #3 = x2
	% #4 = y1
	% #5 = y2
	% #6 = text
	% #7 = box color
	% #8 = border + text formatting
	% #9 = opacity
	\fill[#7,#9,line width=.5cm] (\ox + #2, \oy +  #4) rectangle (\ox + #3, \oy +  #5);
	\draw[#7,#1] (\ox + #2, \oy +  #4) rectangle (\ox + #3, \oy +  #5);
	\if \DrawTriggerLine 1
		\if \DrawTriggerLineBottom 1
			\def \dcoTriggerY{#4}
		\else
			\def \dcoTriggerY{#5}
		\fi
		\draw[#7,#1,c_proc_time,-] (\ox + #2, \oy +  \dcoTriggerY) -- (\ox + #3, \oy +  \dcoTriggerY);
	\fi
	\contourlength{\OutputContour}
	\node[#8,font=\bf\sffamily\normalsize] at ({\ox + #2 + ((#3) - (#2)) / 2}, \oy +  #5 - .25) {\contour{#8!25!black}{#6}}; 
	\if \ShowTiming 1
		\if \DisableTiming 0
			\contourlength{\OutputContour}
			\node[#8,font=\bf\sffamily\scriptsize,c_proc_time] at ({\ox + #2 + (#3 - #2) / 2}, \oy +  #5 - .5) {\contour{#8!25!black}{\Timing}}; 
		\fi
	\fi
}

\def \ShowLateness{0}
\def \latey{.1}
\newcommand{\DrawLatenessHorizonText}[3]{
		\contourlength{0.66pt}
		\node[font=\bf\sffamily\scriptsize,c_outlined_text] at (\ox + #2 + \lateness, \oy +  #3 +.52) {\contour{c_outlined_text_outline}{[12:0#1, 12:0#2)}};
		\node[font=\bf\sffamily\scriptsize,c_outlined_text] at (\ox + #2 + \lateness, \oy +  #3 +.3) {\contour{c_outlined_text_outline}{Lateness Horizon}};
}

\newdimen \lateXDim
\newdimen \maxXDim

\newcommand {\DrawState}[6][\ostate] {
	\def \DrawTriggerLine{0}
	\DrawCustomOutput{#2}{#3}{#4}{#5}{#6}{c_state_back}{c_state_text}{opacity=#1} 
	\def \DrawTriggerLine{1}
	\if \ShowLateness 1
	
		\pgfmathparse{#3 + \lateness}
		\lateXDim=\pgfmathresult pt
	
		\maxXDim=\maxx pt

		\ifthenelse{\lengthtest{\lateXDim > \maxXDim}}
			{\def \lateX{\maxx}}
			{\def \lateX{#3 + \lateness}}
	
		\draw [s_timeline] (\ox + #3 + \lateness, \oy +  #5 - \latey) -- (\ox + #3 + \lateness, \oy + #5 + \latey);
		\DrawLatenessHorizonText{#2}{#3}{#5}
		%\DrawCustomOutput{#3}{#3 + 1}{#4}{#5}{}{s_late}{c_state_text}{opacity=#1} 
	\fi
}

\tikzstyle{s_delay_delta}=[color=c_timeline,densely dotted, line width=.6pt]
\tikzstyle{s_delay_delta_end}=[s_delay_delta, line width=.5pt, solid]

\newcommand{\DrawDelayDelta}[3]{
	%\pgfmathparse{#3 + \lateness}
	%\lateXDim=\pgfmathresult pt
	
	%\maxXDim=\maxx pt

	%\ifthenelse{\lengthtest{\lateXDim > \maxXDim}}
	%	{\def \lateX{\maxx}}
	%	{\def \lateX{#3 + \lateness}}
	
	\def \dddX{#1}
	\def \dddYmin{#2}
	\def \dddYmax{#3}
	
	\pgfmathsetmacro{\dddYdelta}{\dddYmax - \dddYmin}
	\pgfmathsetmacro{\dddYdeltaMin}{(\dddYdelta * 60) / 60)}
	\pgfmathsetmacro{\dddYdeltaSec}{Mod(\dddYdelta * 60, 60)}	
	\pgfmathprintnumberto[fixed,int trunc,verbatim]{\dddYdeltaMin}{\dddDeltaMin}
	\pgfmathprintnumberto[fixed,int trunc,verbatim]{\dddYdeltaSec}{\dddDeltaSec}
	\ifthenelse{\equal{\dddDeltaMin}{0}}
		{\def \dddMin{}}
		{\def \dddMin{\dddDeltaMin m\ }}	
	\ifthenelse{\equal{\dddDeltaSec}{0}}
		{\def \dddSec{}}
		{\def \dddSec{\dddDeltaSec s}}	
	\ifthenelse{\equal{\dddDeltaMin}{0} \AND \equal{\dddDeltaSec}{0}}
		{\def \dddMin{0s}}
		{}	

	\pgfmathsetmacro{\dddYmid}{\dddYmin + (\dddYdelta / 2)}

	
	\draw [s_delay_delta] (\ox + \dddX, \oy +  \dddYmin) -- (\ox + \dddX, \oy + \dddYmax);
	\draw [s_delay_delta_end] (\ox + \dddX - \latey, \oy +  \dddYmin) -- (\ox + \dddX + \latey, \oy + \dddYmin);
	\draw [s_delay_delta_end] (\ox + \dddX - \latey, \oy +  \dddYmax) -- (\ox + \dddX + \latey, \oy + \dddYmax);
	\contourlength{0.66pt}
	\node[font=\bf\sffamily\scriptsize,c_outlined_text,anchor=west] at (\ox + \dddX + 0.01, \oy + \dddYmid) {\contour{c_outlined_text_outline}{\dddMin\dddSec\ delay}};

}

\def \DefaultOutputBack{c_out_back}
\newcommand {\DrawOutput}[6][\DefaultOutputBack] {
	\DrawCustomOutput{#2}{#3}{#4}{#5}{#6}{#1}{c_out_text}{opacity=\oout} 
}

\newcommand {\DrawOutputz}[7][\DefaultOutputBack] {
	\def \ShowTiming{1}
	\def \Timing{#7}
	\DrawOutput[#1]{#2}{#3}{#4}{#5}{#6}
	\def \ShowTiming{0}
}

\newcommand {\DrawTrigger}[5] {
	\DrawCustomOutput{#1}{#2}{#3}{#4}{#5\phantom{---------------}}{white,densely dashed, line width=1pt,draw opacity=1}{white,text opacity=1}{opacity=0} 
}

\newcommand{\offx}[1]{#1 - .04}
\newcommand{\offy}[1]{#1 + .04}

\newcommand{\DrawRetraction}[5]{
	\DrawCustomOutput{#1}{#2}{#3}{#4}{-#5    }{c_retraction_back}{c_retraction_text}{opacity=\oretraction}
}

\newcommand {\DrawRetractionz}[6] {
	\def \ShowTiming{1}
	\def \Timing{#6}
	\DrawRetraction{#1}{#2}{#3}{#4}{#5}
	\def \ShowTiming{0}
}

\definecolor{c_unused}{HTML}{123456}
\newcommand{\DrawTombstone}[4]{
	\def \DrawTriggerLine{0}
	\DrawCustomOutput[s_tomb]{#1}{#2}{#3}{#4}{}{c_tomb}{c_unused}{opacity=0}
	\def \DrawTriggerLine{1}
	\draw [s_tomb_late] (\ox + #2, \oy +  #4) -- (\ox + #2 + \lateness, \oy +  #4);
	\draw [s_tomb_late] (\ox + #2 + \lateness, \oy +  #4 - \latey) -- (\ox + #2 + \lateness, \oy + #4 + \latey);
	\DrawLatenessHorizonText{#1}{#2}{#4}
}

\newcommand{\DrawTimeline}[2]{
	\begin{scope}
	% Never draw outside max boundaries
	\clip (\ox, \oy) rectangle (\ox + \maxx, \oy + \maxy);
	\draw [s_timeline,#1] (\ox + 0, \oy +  #2) -- (\ox + \maxx, \oy +  #2);
	\end{scope}
}

\def \ShowWatermarks{1}
\ifdefined \ShowIdealWatermarkOnly
\else
	\def \ShowIdealWatermarkOnly{0}
\fi

\ifdefined \DiagramName
\else
	\def \DiagramName{}
\fi

\ifdefined \BackRectBorderWidth
\else
	\def \BackRectBorderWidth{0pt}
\fi

\ifdefined \BackRectBorderColor
\else
	\def \BackRectBorderColor{c_back_light}
\fi

\newcommand{\BackRect}[0] {
	\if \DrawDefaultLegend 1
		\def \ExtraHeight{1}
	\else
		\def \ExtraHeight{0}
	\fi
	\fill[c_back_dark,draw=\BackRectBorderColor,line width=\BackRectBorderWidth] (\ox - 1.333, \oy - 1 - \ExtraHeight) rectangle (\ox + \maxx + 1, \oy + \maxy + 1);
	\node [c_legend,font=\large] at (\ox + \maxx / 2 - .1666, \oy + \maxy + .5) {\DiagramName};
}

\ifdefined \DrawBackRect
\else
	\def \DrawBackRect{0}
\fi

\newcommand{\GamingInline}[4][c_timeline] {
	\if \DrawBackRect 1	
		\BackRect
	\fi
	\if \ShowWatermarks 1
		\DrawWatermarks{#2} 
	\fi
	\if \ShowIdealWatermarkOnly 1
		\DrawIdealWatermarkLegend{-1.15}
		\DrawIdealWatermark
	\fi
	\begin{scope}
		\clip (\ox + 0.01, \oy) rectangle (\ox + \maxx, \oy +  #2);
		\DrawInputs
	\end{scope}
	\begin{scope}
		\clip (\ox + 0.01, \oy +  5) rectangle (\ox + \maxx, \oy +  #2);
		\DrawGhostInputs
	\end{scope}
	#3
	#4
	\DrawTimeline{#1}{#2}
}

\ifdefined \XAxisToTen
\else
	\def \XAxisToTen{0}
\fi

\ifdefined \YAxisToEleven
\else
	\def \YAxisToEleven{0}
\fi

\ifdefined \OutputOverTimeline
\else
	\def \OutputOverTimeline{0}
\fi

\ifdefined \ClipInputs
\else
	\def \ClipInputs{0}
\fi

\newcommand{\TriggersInline}[3][c_timeline] {
	\if \DrawBackRect 1	
		\BackRect
	\fi
	\if \ShowWatermarks 1
		\DrawWatermarks{#2} 
	\fi
	\if \ShowIdealWatermarkOnly 1
		\DrawIdealWatermarkLegend{-1.15}
		\DrawIdealWatermark
	\fi
	\begin{scope}
		\if \ClipInputs 1
			\clip (\ox + 0.01, \oy) rectangle (\ox + \maxx, \oy +  #2);
		\fi
		\DrawInputs
	\end{scope}
	\begin{scope}
		\clip (\ox + 0.01, \oy +  5) rectangle (\ox + \maxx, \oy +  #2);
		\DrawGhostInputs
	\end{scope}
	\DrawAxes{\XAxisToTen} 

	\if \OutputOverTimeline 0
		#3
	\fi
	\DrawTimeline{#1}{#2}
	\if \OutputOverTimeline 1
		#3
	\fi
}

\ifdefined \ShowFrame
\else
	\def \ShowFrame{0}
\fi

\ifdefined \frame
\else
	\def \frame{n/a}
\fi

\ifdefined \FrameDebugX
\else
	\def \FrameDebugX{-2}
\fi

\ifdefined \FrameDebugY
\else
	\def \FrameDebugY{-0.666}
\fi

\newcommand{\TriggersMulti}[1] {
\TikzDiagram[7cm]{
	\if \ShowFrame 1
		\node[fill=c_back_dark,draw=c_red,text=c_red,rectangle,font=\scriptsize,solid] at (\FrameDebugX, \FrameDebugY) {\frame : \offset};
	\fi
  #1
}
}

\newcommand{\BoundingBox}[7] {
        % #1 = x
        % #2 = y
        % #3 = w
        % #4 = h
        % #5 = fill opacity
        % #6 = draw opacity
        % #7 = bordercolor
        \def \px{#1}
        \def \py{#2}
        \def \pw{#3}
        \def \ph{#4}
        \def \pFillOpacity{#5}
        \def \pDrawOpacity{#6}
        \def \pline{#7}
	\fill[fill=c_back_dark,draw=\pline,line width=1pt,densely dashed,opacity=\pFillOpacity,draw opacity=\pDrawOpacity] (\px, \py) rectangle (\px + \pw, \py + \ph);
}


\newdimen \dimYMinBase
\newdimen \dimZero

\newcommand {\DrawYAxis}[2] {
	\def \dyaX{#1}
	\pgfmathprintnumberto[fixed,int trunc,verbatim]{\stYMinBase}{\stYMinBaseInt}
	\pgfmathsetmacro{\stYMinBaseMod}{\stYMinBase - \stYMinBaseInt}
	\pgfmathsetmacro{\stYMinBase}{\stYMinBase}
	\dimYMinBase=\stYMinBase pt
	\dimZero=0 pt
	\ifthenelse{\lengthtest{\dimYMinBase < \dimZero}}
		{
			\pgfmathsetmacro{\stYMinBaseMod}{-\stYMinBaseMod}
			\def \dyaExtendDown{1}
		}
		{\def \dyaExtendDown{0}}
	\pgfmathsetmacro{\dyaY}{\stInputY + \stYMinBaseMod}
	\pgfmathsetmacro{\stYMinBase}{\stYMinBaseInt}
	\def \dyaLabelOnLeft{#2}
	\def \dyaLabelOffset{0.25}
	\def \dyaLabelRotate{-90}
	\if \dyaLabelOnLeft 1
		\pgfmathsetmacro{\dyaLabelOffset}{-\dyaLabelOffset}
		%\pgfmathsetmacro{\dyaLabelRotate}{-\dyaLabelRotate}
	\fi
	\def \dyaH{20}
	% Y Axis
	\if \dyaExtendDown 1
		\draw[c_proc_time, line width=1.5pt] (\dyaX, \dyaY - 0.0265 - \stYMinBaseMod) -- (\dyaX, \dyaY + \dyaH);
	\else
		\draw[c_proc_time, line width=1.5pt] (\dyaX, \dyaY - 0.0265) -- (\dyaX, \dyaY + \dyaH + \stYMinBaseMod);
	\fi	
	% Y Axis Tick Marks
	\foreach \y in {0,...,\dyaH} {
		\draw[c_proc_time] (\dyaX + .1, \dyaY + \y) -- (\dyaX - .1, \dyaY + \y);
		\pgfmathsetmacro{\dyaMinBase}{\y + \stYMinBase}
		\pgfmathsetmacro{\dyaHour}{\dyaMinBase<0?"11":"12"}
		\pgfmathsetmacro{\dyaMinBase}{\dyaMinBase<0?\dyaMinBase+60:\dyaMinBase}
		\pgfmathsetmacro{\dyaMin}{mod(\dyaMinBase,60)<10?"0":{},int(mod(\dyaMinBase,60))}
		\node[s_timepoint, rotate=\dyaLabelRotate, c_proc_time] at (\dyaX + \dyaLabelOffset, \dyaY + \y) {\dyaHour:\dyaMin};
	}	
	% Y Axis Label
	%\node[s_legend,rotate=90, c_proc_time] at (\dyaX -.666, \dyaY + \stH / 2) {Processing Time};	
	
	% Corner beautification
	%\fill[c_event_time, line width=0pt] (\dyaX - \trioff, \dyaY - \trioff) -- (\dyaX + \trioff, \dyaY + \trioff) -- (\dyaX + \trioff, \dyaY - \trioff) -- (\dyaX - \trioff, \dyaY - \trioff);
}


\newcommand {\DrawYAxisOrig}[2] {
	\def \dyaX{#1}
	\def \dyaLabelOnLeft{#2}
	\def \dyaLabelOffset{0.25}
	\def \dyaLabelRotate{-90}
	\if \dyaLabelOnLeft 1
		\pgfmathsetmacro{\dyaLabelOffset}{-\dyaLabelOffset}
		%\pgfmathsetmacro{\dyaLabelRotate}{-\dyaLabelRotate}
	\fi
	\def \dyaH{20}
	% Y Axis
	\draw[c_proc_time, line width=1.5pt] (\dyaX, \stInputY - 0.0265) -- (\dyaX, \stInputY + \dyaH);
	% Y Axis Tick Marks
	\foreach \y in {0,...,\dyaH} {
		\draw[c_proc_time] (\dyaX + .1, \stInputY + \y) -- (\dyaX - .1, \stInputY + \y);
		\pgfmathsetmacro{\dyaMinBase}{\y + \stYMinBase}
		\pgfmathsetmacro{\dyaHour}{\dyaMinBase<0?"11":"12"}
		\pgfmathsetmacro{\dyaMinBase}{\dyaMinBase<0?\dyaMinBase+60:\dyaMinBase}
		\pgfmathsetmacro{\dyaMin}{mod(\dyaMinBase,60)<10?"0":{},int(mod(\dyaMinBase,60))}
		\node[s_timepoint, rotate=\dyaLabelRotate, c_proc_time] at (\dyaX + \dyaLabelOffset, \stInputY + \y) {\dyaHour:\dyaMin};
	}	
	% Y Axis Label
	%\node[s_legend,rotate=90, c_proc_time] at (\dyaX -.666, \stInputY + \stH / 2) {Processing Time};	
	
	% Corner beautification
	%\fill[c_event_time, line width=0pt] (\dyaX - \trioff, \stInputY - \trioff) -- (\dyaX + \trioff, \stInputY + \trioff) -- (\dyaX + \trioff, \stInputY - \trioff) -- (\dyaX - \trioff, \stInputY - \trioff);
}

\newcommand{\DrawXAxis}[2] {
	\def \dxaY{#1}
	\def \dxaLabelOnTop{#2}
	\if \dxaLabelOnTop 1
		\pgfmathsetmacro{\dxaLabelY}{\dxaY + 0.25}
	\else
		\pgfmathsetmacro{\dxaLabelY}{\dxaY - 0.25}
		% X Axis Label
		\node[s_legend,color=c_event_time] at (\stX + \stW / 2, \dxaY - 0.75) {Event Time};
	\fi
	% X Axis
	\draw[color=c_event_time, line width=1.5pt] (\stX,\dxaY) -- (\stXW, \dxaY);
	% X Axis Tick Marks
	\foreach \x in {0,...,\stXW} {
		\draw[color=c_event_time] (\stX + \x, \dxaY + .1) -- (\stX + \x, \dxaY - .1);
		\pgfmathsetmacro{\dxaMin}{mod(\x,60)<10?"0":{},int(mod(\x,60))}
		\node[s_timepoint,color=c_event_time] at (\stX + \x, \dxaLabelY) {12:\dxaMin};
	}
}

\newcommand{\DrawDivider}[4][dotted] {
	\def \ddY{#2}
	\def \ddExtraLeft{#3}
	\def \ddExtraRight{#4}
	\def \ddLineStyle{#1}
	\draw[color=c_state_back, line width=0.5pt, \ddLineStyle] (\stX - \ddExtraLeft, \ddY) -- (\stXW + \ddExtraRight, \ddY);
}

\newcommand{\DrawTableDivider}[1] {
	\DrawDivider{#1}{3.5}{3.5}
}

\newcommand{\DrawEndOfStream}[1] {
	\tikzstyle s_deoi_line=[densely dotted,color=c_proc_time]
	\def \deoiYoff{#1}
	\def \deoiY{\oy + \deoiYoff}
	\DrawDivider[s_deoi_line]{\deoiY}{0}{0}
	\node[fill=c_back_dark,draw=c_proc_time,text=c_proc_time,rectangle,font=\scriptsize,densely dotted] at (\stX + \stW / 2, \deoiY) {END OF INPUT};
}

\newcommand{\DrawBytesRow}[2] {
	\def \dbrYoff{#1}
	\def \dbrText{#2}
	\contourlength{0.4pt} 
	\node[font=\bf\ttfamily\tiny,thick,draw opacity=0,fill opacity=0,text opacity=1,c_datum_text] at (\dbX, \dbY + \dbrYoff) {\contour{c_datum_border}{\dbrText}};
}

\newcommand{\DrawBytes}[6][c_datum_border] {
	\def \dbBackColor{#1}
	\def \dbX{#2}
	\def \dbY{#3 + \stInputY + 2}
	\def \dbTextA{#4}
	\def \dbTextB{#5}
	\def \dbTextC{#6}
	\def \dbW{.25}
	\def \dbH{.26}
	\filldraw[line width=0.333pt,\dbBackColor,draw=c_datum_back] (\dbX - \dbW, \dbY - \dbH) rectangle (\dbX + \dbW, \dbY + \dbH);
	\DrawBytesRow{0.15}{\dbTextA}
	\DrawBytesRow{0}{\dbTextB}
	\DrawBytesRow{-0.15}{\dbTextC}
}

\tikzstyle{s_bytes}=[fill=c_datum_back] 
\tikzstyle{s_bytes_dropped}=[fill=red!60!white,draw opacity=.25] 

\newcommand{\DrawBytesA} {
	\DrawBytes{0.444}{.333}{0100}{1101}{1010}
	\DrawBytes{2.444}{.666}{1000}{0101}{0001}
	\DrawBytes{3.111}{2.111 + \InputsAEightBump}{0110}{0011}{1101}
	\DrawBytes{3.666}{1.222}{1011}{0100}{0100}
	%\DrawBytes{3.888}{1.444}{1101}{1000}{0000}
	\DrawBytes{4.333}{1.666 + \InputsAFourBump}{0001}{1110}{1000}
	\DrawBytes{6.666}{2.333}{1010}{1111}{0110}
	\DrawBytes{7.444}{3.666 - \InputsAEightBump}{1000}{0001}{0001}
	\DrawBytes{7.777}{4.000}{0000}{0100}{0000}
	\if \NineIsLate 1
		\DrawBytes[s_bytes_dropped]{1.444}{3.333}{0101}{1100}{1110}
	\else
		\DrawBytes{1.444}{3.333}{0101}{1100}{1110}
	\fi
}

\newcommand {\DrawCurrency}[5][0] {
	\def \dcW{#1}
	\def \dcX{#2}
	\def \dcY{#3}
	\colorlet{c_dc_text_color}{#4}
	\def \dcText{#5}
	\if \dcW 0
		\filldraw[c_bullet_fill, draw=c_bullet_draw] (\ox + \dcX, \oy + \dcY) circle [radius=2pt];
	\else
		\fill[c_state_back,opacity=\ostate,line width=.5cm] (\ox + \dcX, \oy +  \dcY - 0.25) rectangle (\ox + \dcX + \dcW, \oy +  \dcY + 0.25);
		\draw[c_bullet_fill] (\ox + \dcX, \oy +  \dcY - 0.25) rectangle (\ox + \dcX + \dcW, \oy +  \dcY + 0.25);
	\fi
	\contourlength{1pt}
	\node[c_dc_text_color,font=\bf\sffamily\normalsize,text width=2cm] at ({\ox + \dcX + 1 + .15}, \oy + \dcY) {\contour{c_state_text!25!black}{\dcText}}; 	
}

\newcommand{\DrawTableCell}[4][\ostate] {
	\def \dtXstart{#2}
	\def \dtXend{#3}
	\def \dtText{#4}
	% DrawTableCell's Y coordinates are given relative to the viewport, not
	% the origin, so we have to subtract out the origin.
	\DrawState[#1]{\dtXstart}{\dtXend}{\stY - \oy}{\stY + \stH - \oy}{\dtText}
}

\newcommand{\StreamTableInline}[2] {
	% Requires the following macros be predefined as input parameters
	% \stX, \stY, \stW, \stH, \stYoff, \stDrawYAxis, \stYMinBase & \stYAxisLabel (both only if \stDrawYAxis is 1),
	% \stLabel
        \def \stDebugColor{#1}
        \pgfmathsetmacro{\stXW}{\stX + \stW}
        \pgfmathsetmacro{\stYH}{\stY + \stH}
        \pgfmathsetmacro{\stYShift}{\oyBase - \stY}
        \def \stDrawingCommands{#2}
        \pgfmathsetmacro{\stInputY}{\stY - \stYoff}
        \pgfmathsetmacro{\oy}{\oyBase - \stYoff}
        	\pgfmathsetmacro{\stSecTmp}{int(mod(\stYoff, 1) * 60)}
	\pgfmathsetmacro{\stSec}{mod(\stSecTmp,60)<10?"0":{},int(mod(\stSecTmp,60))}
	\pgfmathsetmacro{\stMinTmp}{int(\stYMinBase + int(\stYoff))}
	\pgfmathsetmacro{\stMin}{mod(\stMinTmp,60)<10?"0":{},int(mod(\stMinTmp,60))}
	%\pgfmathprintnumberto[
	%    fixed,
	%    precision=0,
    	%    verbatim,
	%]{\stSecTmp}{\stSec}
	%\BoundingBox{\stX}{\stY}{\stW}{\stH}{1}{0}{c_blue}
		
	% Main bounding box
	\if \DebugStreamTable 1	
	\else
		\begin{scope}
		\clip (\stX, \stY - \stYBoundBoostDown) rectangle (\stXW, \stYH + \stYBoundBoost);
	\fi

	\if \stDrawYAxis 1
	\else
		\DrawTableCell[0]{\stX}{\stXW}{}
	\fi

	\stDrawingCommands

	% Left margin
	\if \DebugStreamTable 1
	\else
		\end{scope}
		\begin{scope}
		\clip (\stX - 3.5, \stY) rectangle (\stX, \stYH);
	\fi	
	\if \stDrawYAxis 1
		\DrawYAxis{\stX}{1}
		\if \stYAxisLabel 1
			\node[s_legend,rotate=-90, c_proc_time] at (\stX - 0.75, \stY + \stH / 2) {Processing Time};	
		\fi
		\node[s_margin_data] at (\stX - 2.25, \stY + \stH / 2) {\stLabel};		
	\else
		% The 69 is to make sure the height of the clock stays constant even with the
		% Fira Sans font, which has a variable amount of upward and downward swoop.
		% We block it out with a clip.
		\begin{scope}
			\clip (\stX - 1.17, \stY - \stH) rectangle (\stX + 2, \stY + \stH);
			\node[s_timepoint, rotate=0, c_proc_time, anchor=west] at (\stX - 1.6, \stY + \stH / 2) {69 12:\stMin:\stSec};
		\end{scope}
		\node[s_margin_op] at (\stX - 2.25, \stY + \stH / 2) {\stLabel};
	\fi

	% Right margin
	\if \DebugStreamTable 1
	\else
		\end{scope}
		\begin{scope}
		\clip (\stXW, \stY) rectangle (\stXW + 3.5, \stYH);
	\fi	
	\if \stDrawYAxis 1
		\DrawYAxis{\stXW}{0}
		\node[s_margin_op] at (\stXW + 2.25, \stY + \stH / 2) {STREAM};
	\else
		% The 69 is to make sure the height of the clock stays constant even with the
		% Fira Sans font, which has a variable amount of upward and downward swoop.
		% We block it out with a clip.
		\begin{scope}
			\clip (\stXW - 0.3, \stY - \stH) rectangle (\stXW + 1.17, \stY + \stH);
			\node[s_timepoint, rotate=0, c_proc_time, anchor=west] at (\stXW + 0.05, \stY + \stH / 2) {12:\stMin:\stSec\ \ 69};
		\end{scope}
		\node[s_margin_op] at (\stXW + 2.25, \stY + \stH / 2) {TABLE};
	\fi
	
	\if \DebugStreamTable 1
	\else
		\end{scope}
	\fi
	
	% Table dividers
	\if \stDrawYAxis 1
	\else
		\DrawTableDivider{2}
		\DrawTableDivider{2.5}
	\fi
	
}


\newcommand{\StreamInline}[1] {
	\def \stDrawYAxis{1}
	\StreamTableInline{c_blue}{#1}
}

\newcommand{\TableInline}[6] {
	\def \stDrawYAxis{0}
	\def \stYAxisLabel{0}
	\StreamTableInline{c_green}{#1}
}

\newcommand{\RenderFrames} {
	\if \FastRender 0
		\RenderFramesImpl{\AllFrames}
	\else
		\RenderFramesImpl{\FastFrames}
	\fi
}

\newcommand{\RenderFramesImpl}[1] {
\foreach \frame [evaluate=\frame as \offsetTmp using \frame/9] in #1 {
	\pgfmathprintnumberto[fixed,precision=3,verbatim]{\offsetTmp}{\offset}
	\pgfmathprintnumberto[fixed,int trunc,verbatim]{\offsetTmp}{\offsetInt}
	\pgfmathsetmacro{\offsetDecimal}{\offset - \offsetInt}
	\ifthenelse{\equal{\offsetDecimal}{0.556}
			\OR \equal{\offsetDecimal}{0.667}
			\OR \equal{\offsetDecimal}{0.778}
			\OR \equal{\offsetDecimal}{0.889}}
		{\pgfmathsetmacro{\offset}{\offset - 0.001}}
		{}
	\Frame{\offset}
}
}

\newcommand {\GamingDiagram}[4][c_timeline] {
\TriggersMulti{
	\GamingInline[#1]{#2}{#3}{#4}
}
}

\newcommand {\TriggersDiagram}[3][c_timeline] {
\TriggersMulti{
	\TriggersInline[#1]{#2}{#3}
}
}

\ifdefined \FastRender
\else
	\def \FastRender{0}
\fi

\newcommand{\DrawInitialFrame}[1] {
        \if \FastRender 0
		#1
        \fi
}

\newcommand{\DrawFinalFrames}[2][19] {
        #2
        \if \FastRender 0
	\foreach \i in {2,...,#1} {
		#2
	}
        \fi
	
}

\ifdefined \JointLeftTitle
\else
	\def \JointLeftTitle{Ordering \#1}
\fi

\ifdefined \JointLeftPerfectWatermark
\else
	\def \JointLeftPerfectWatermark{0}
\fi

\newcommand{\JointLeft}[1]{
  \if \JointLeftPerfectWatermark 1
	\def \UsePerfectWatermark{1}
	\def \UseHeuristicWatermark{0}
  \else
	\def \UsePerfectWatermark{0}
	\def \UseHeuristicWatermark{1}
  \fi
  \def \InputVersion{\JointLeftInput}
  \def \ox{0}
  \def \oy{0}
  \def \DiagramName{\JointLeftTitle}
  #1
}

\ifdefined \JointRightTitle
\else
	\def \JointRightTitle{Ordering \#2}
\fi

\ifdefined \JointRightNineIsLate
\else
	\def \JointRightNineIsLate{0}
\fi

\ifdefined \JointRightPerfectWatermark
\else
	\def \JointRightPerfectWatermark{0}
\fi

\newcommand{\JointRight}[1]{
  \if \JointRightPerfectWatermark 1
	\def \UsePerfectWatermark{1}
	\def \UseHeuristicWatermark{0}
  \else
	\def \UsePerfectWatermark{0}
	\def \UseHeuristicWatermark{1}
  \fi
  \def \InputVersion{\JointRightInput}
  %\def \ox{0}
  %\def \oy{-\maxy / 2 - 1}
 % \def \ox{\maxx + 2.55}
  %\def \ox{\maxx + 2.3} % Overlapping
  %\def \ox{\maxx + 2.65}
  \def \ox{\maxx + 2.0}%2.69}
  \def \oy{0}
  \def \DiagramName{\JointRightTitle}
  \def \NineIsLate{\JointRightNineIsLate}
  #1
}
	
\newcommand{\JointFrame}[2] {
\TriggersMulti{

  \JointLeft{#1}
  \JointRight{#2}

  \def \wmx{\maxx - .1} 
  \def \wmy{-1.4}
  \def \xa{\maxx + .5 - 1.39} % 2.55
  \def \xa{\maxx + .5 - 1.32} % 2.69
  \def \wa{4}
  \def \xb{\xa + \wa}
  \def \xm{{\xa + (\wa / 2)}}

  % Measuring guides
  %\draw [-, s_watermark, color=white] (\xa, -1.2) -- (\xb, -1.2);
  %\draw [-, s_watermark, color=white] (\xm, -1.25) -- (\xm, -1.15);
  \if \ShowJointLegend 1
  	\if \UsePerfectWatermark 1
		\WatermarkLegend[-.5]{\wmx}{\wmy}{Perfect Watermark}{s_watermark,c_perfect_watermark}
	\else
		\WatermarkLegend[-.5]{\wmx}{\wmy}{Heuristic Watermark}{s_watermark}
	\fi
	\WatermarkLegend[-.5]{\wmx}{\wmy - .4}{Ideal Watermark}{s_ideal}
  \fi
}
}

\newcommand{\SetJointVars}[2] {
%\def \ShowJointLegend{#1}
\def \ShowJointLegend{0}
\def \ShowLateData{#2}
\def \ShowAlignmentHelpers{0}
\def \DrawDefaultLegend{#1}
\def \BackRectBorderColor{c_back_dark}
\def \BackRectBorderWidth{0pt}
\def \DrawBackRect{0}
%\def \BackBorder{c_back}
\def \BackBorder{c_back_dark}
%\def \BackBorderWidth{2ex}
\def \BackBorderWidth{0pt}
\def \DrawBackRect{1}
\def \ShowWatermarks{#1}
}

\newdimen \targetY
\newdimen \currentY

\newcommand{\DrawAt}[3]{
  % #1 = current y
  % #2 = target y
  % #3 = commands
  \currentY=#1 pt
  \targetY=#2 pt
  \ifthenelse{\lengthtest{\targetY > \currentY}}
    {}
    {#3}
}

\newdimen \currentY
\newdimen \startY
\newdimen \endY

\newcommand{\DrawIn}[4]{
  % #1 = current y
  % #2 = start y (inclusive)
  % #3 = end y (inclusive)
  % #4 = commands
  \ifdefined \diOffset
  \else
  	\def \diOffset{0}
  \fi
  \pgfmathsetmacro{\diCurrent}{#1}
  \pgfmathsetmacro{\diStart}{#2+\diOffset}
  \pgfmathsetmacro{\diEnd}{#3+\diOffset}
  \currentY=\diCurrent pt
  \startY=\diStart pt
  \endY=\diEnd pt
  \ifthenelse{\lengthtest{\startY > \currentY}}
    {\def \AfterStart{0}}
    {\def \AfterStart{1}}
   \ifthenelse{\lengthtest{\currentY > \endY}}
    {\def \BeforeEnd{0}} % after
    {\def \BeforeEnd{1}}
    \if \AfterStart 1
    	\if \BeforeEnd 1
		#4
	\fi
    \fi
}

\newcommand{\DrawWindowedRate}[4] {
	\def \dwrXstart{#1}
	\def \dwrXend{#2}
	\def \dwrY{#3}
	\def \dwrText{#4}
	\def \dwrDelta{.25}
	\DrawState[\ostate]{\dwrXstart}{\dwrXend}{\dwrY - \dwrDelta}{\dwrY + \dwrDelta}{\dwrText}
}

\newcommand{\DrawWindowedOutputz}[5][\DefaultOutputBack]{
	\def \proBack{#1}
	\def \proY{#4}
	\def \proDelta{.25}
	\def \proXA{#2}
	\def \proXB{#3}
	\def \proVal{#5}
	%\DrawIn{\stYoff}{\proYBottom}{12.0}{
	\DrawOutputz[\proBack]{\proXA}{\proXB}{\proY - \proDelta}{\proY + \proDelta}{\proVal}{}
	%}
}

\newcommand{\DrawTableOutputz}[5][\DefaultOutputBack]{
	\def \proBack{#1}
	\def \proY{#2}
	\def \proYBottom{#2 - 0.5}
	\def \proXA{#3}
	\def \proXB{#4}
	\def \proVal{#5}
	\DrawIn{\stYoff}{\proY}{12.0}{\DrawOutputz[\proBack]{\proXA}{\proXB}{\proYBottom}{\proY}{\proVal}{}}
}
